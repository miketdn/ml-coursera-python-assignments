
% Default to the notebook output style

    


% Inherit from the specified cell style.




    
\documentclass[11pt]{article}

    
    
    \usepackage[T1]{fontenc}
    % Nicer default font (+ math font) than Computer Modern for most use cases
    \usepackage{mathpazo}

    % Basic figure setup, for now with no caption control since it's done
    % automatically by Pandoc (which extracts ![](path) syntax from Markdown).
    \usepackage{graphicx}
    % We will generate all images so they have a width \maxwidth. This means
    % that they will get their normal width if they fit onto the page, but
    % are scaled down if they would overflow the margins.
    \makeatletter
    \def\maxwidth{\ifdim\Gin@nat@width>\linewidth\linewidth
    \else\Gin@nat@width\fi}
    \makeatother
    \let\Oldincludegraphics\includegraphics
    % Set max figure width to be 80% of text width, for now hardcoded.
    \renewcommand{\includegraphics}[1]{\Oldincludegraphics[width=.8\maxwidth]{#1}}
    % Ensure that by default, figures have no caption (until we provide a
    % proper Figure object with a Caption API and a way to capture that
    % in the conversion process - todo).
    \usepackage{caption}
    \DeclareCaptionLabelFormat{nolabel}{}
    \captionsetup{labelformat=nolabel}

    \usepackage{adjustbox} % Used to constrain images to a maximum size 
    \usepackage{xcolor} % Allow colors to be defined
    \usepackage{enumerate} % Needed for markdown enumerations to work
    \usepackage{geometry} % Used to adjust the document margins
    \usepackage{amsmath} % Equations
    \usepackage{amssymb} % Equations
    \usepackage{textcomp} % defines textquotesingle
    % Hack from http://tex.stackexchange.com/a/47451/13684:
    \AtBeginDocument{%
        \def\PYZsq{\textquotesingle}% Upright quotes in Pygmentized code
    }
    \usepackage{upquote} % Upright quotes for verbatim code
    \usepackage{eurosym} % defines \euro
    \usepackage[mathletters]{ucs} % Extended unicode (utf-8) support
    \usepackage[utf8x]{inputenc} % Allow utf-8 characters in the tex document
    \usepackage{fancyvrb} % verbatim replacement that allows latex
    \usepackage{grffile} % extends the file name processing of package graphics 
                         % to support a larger range 
    % The hyperref package gives us a pdf with properly built
    % internal navigation ('pdf bookmarks' for the table of contents,
    % internal cross-reference links, web links for URLs, etc.)
    \usepackage{hyperref}
    \usepackage{longtable} % longtable support required by pandoc >1.10
    \usepackage{booktabs}  % table support for pandoc > 1.12.2
    \usepackage[inline]{enumitem} % IRkernel/repr support (it uses the enumerate* environment)
    \usepackage[normalem]{ulem} % ulem is needed to support strikethroughs (\sout)
                                % normalem makes italics be italics, not underlines
    \usepackage{mathrsfs}
    

    
    
    % Colors for the hyperref package
    \definecolor{urlcolor}{rgb}{0,.145,.698}
    \definecolor{linkcolor}{rgb}{.71,0.21,0.01}
    \definecolor{citecolor}{rgb}{.12,.54,.11}

    % ANSI colors
    \definecolor{ansi-black}{HTML}{3E424D}
    \definecolor{ansi-black-intense}{HTML}{282C36}
    \definecolor{ansi-red}{HTML}{E75C58}
    \definecolor{ansi-red-intense}{HTML}{B22B31}
    \definecolor{ansi-green}{HTML}{00A250}
    \definecolor{ansi-green-intense}{HTML}{007427}
    \definecolor{ansi-yellow}{HTML}{DDB62B}
    \definecolor{ansi-yellow-intense}{HTML}{B27D12}
    \definecolor{ansi-blue}{HTML}{208FFB}
    \definecolor{ansi-blue-intense}{HTML}{0065CA}
    \definecolor{ansi-magenta}{HTML}{D160C4}
    \definecolor{ansi-magenta-intense}{HTML}{A03196}
    \definecolor{ansi-cyan}{HTML}{60C6C8}
    \definecolor{ansi-cyan-intense}{HTML}{258F8F}
    \definecolor{ansi-white}{HTML}{C5C1B4}
    \definecolor{ansi-white-intense}{HTML}{A1A6B2}
    \definecolor{ansi-default-inverse-fg}{HTML}{FFFFFF}
    \definecolor{ansi-default-inverse-bg}{HTML}{000000}

    % commands and environments needed by pandoc snippets
    % extracted from the output of `pandoc -s`
    \providecommand{\tightlist}{%
      \setlength{\itemsep}{0pt}\setlength{\parskip}{0pt}}
    \DefineVerbatimEnvironment{Highlighting}{Verbatim}{commandchars=\\\{\}}
    % Add ',fontsize=\small' for more characters per line
    \newenvironment{Shaded}{}{}
    \newcommand{\KeywordTok}[1]{\textcolor[rgb]{0.00,0.44,0.13}{\textbf{{#1}}}}
    \newcommand{\DataTypeTok}[1]{\textcolor[rgb]{0.56,0.13,0.00}{{#1}}}
    \newcommand{\DecValTok}[1]{\textcolor[rgb]{0.25,0.63,0.44}{{#1}}}
    \newcommand{\BaseNTok}[1]{\textcolor[rgb]{0.25,0.63,0.44}{{#1}}}
    \newcommand{\FloatTok}[1]{\textcolor[rgb]{0.25,0.63,0.44}{{#1}}}
    \newcommand{\CharTok}[1]{\textcolor[rgb]{0.25,0.44,0.63}{{#1}}}
    \newcommand{\StringTok}[1]{\textcolor[rgb]{0.25,0.44,0.63}{{#1}}}
    \newcommand{\CommentTok}[1]{\textcolor[rgb]{0.38,0.63,0.69}{\textit{{#1}}}}
    \newcommand{\OtherTok}[1]{\textcolor[rgb]{0.00,0.44,0.13}{{#1}}}
    \newcommand{\AlertTok}[1]{\textcolor[rgb]{1.00,0.00,0.00}{\textbf{{#1}}}}
    \newcommand{\FunctionTok}[1]{\textcolor[rgb]{0.02,0.16,0.49}{{#1}}}
    \newcommand{\RegionMarkerTok}[1]{{#1}}
    \newcommand{\ErrorTok}[1]{\textcolor[rgb]{1.00,0.00,0.00}{\textbf{{#1}}}}
    \newcommand{\NormalTok}[1]{{#1}}
    
    % Additional commands for more recent versions of Pandoc
    \newcommand{\ConstantTok}[1]{\textcolor[rgb]{0.53,0.00,0.00}{{#1}}}
    \newcommand{\SpecialCharTok}[1]{\textcolor[rgb]{0.25,0.44,0.63}{{#1}}}
    \newcommand{\VerbatimStringTok}[1]{\textcolor[rgb]{0.25,0.44,0.63}{{#1}}}
    \newcommand{\SpecialStringTok}[1]{\textcolor[rgb]{0.73,0.40,0.53}{{#1}}}
    \newcommand{\ImportTok}[1]{{#1}}
    \newcommand{\DocumentationTok}[1]{\textcolor[rgb]{0.73,0.13,0.13}{\textit{{#1}}}}
    \newcommand{\AnnotationTok}[1]{\textcolor[rgb]{0.38,0.63,0.69}{\textbf{\textit{{#1}}}}}
    \newcommand{\CommentVarTok}[1]{\textcolor[rgb]{0.38,0.63,0.69}{\textbf{\textit{{#1}}}}}
    \newcommand{\VariableTok}[1]{\textcolor[rgb]{0.10,0.09,0.49}{{#1}}}
    \newcommand{\ControlFlowTok}[1]{\textcolor[rgb]{0.00,0.44,0.13}{\textbf{{#1}}}}
    \newcommand{\OperatorTok}[1]{\textcolor[rgb]{0.40,0.40,0.40}{{#1}}}
    \newcommand{\BuiltInTok}[1]{{#1}}
    \newcommand{\ExtensionTok}[1]{{#1}}
    \newcommand{\PreprocessorTok}[1]{\textcolor[rgb]{0.74,0.48,0.00}{{#1}}}
    \newcommand{\AttributeTok}[1]{\textcolor[rgb]{0.49,0.56,0.16}{{#1}}}
    \newcommand{\InformationTok}[1]{\textcolor[rgb]{0.38,0.63,0.69}{\textbf{\textit{{#1}}}}}
    \newcommand{\WarningTok}[1]{\textcolor[rgb]{0.38,0.63,0.69}{\textbf{\textit{{#1}}}}}
    
    
    % Define a nice break command that doesn't care if a line doesn't already
    % exist.
    \def\br{\hspace*{\fill} \\* }
    % Math Jax compatibility definitions
    \def\gt{>}
    \def\lt{<}
    \let\Oldtex\TeX
    \let\Oldlatex\LaTeX
    \renewcommand{\TeX}{\textrm{\Oldtex}}
    \renewcommand{\LaTeX}{\textrm{\Oldlatex}}
    % Document parameters
    % Document title
    \title{exercise1}
    
    
    
    
    

    % Pygments definitions
    
\makeatletter
\def\PY@reset{\let\PY@it=\relax \let\PY@bf=\relax%
    \let\PY@ul=\relax \let\PY@tc=\relax%
    \let\PY@bc=\relax \let\PY@ff=\relax}
\def\PY@tok#1{\csname PY@tok@#1\endcsname}
\def\PY@toks#1+{\ifx\relax#1\empty\else%
    \PY@tok{#1}\expandafter\PY@toks\fi}
\def\PY@do#1{\PY@bc{\PY@tc{\PY@ul{%
    \PY@it{\PY@bf{\PY@ff{#1}}}}}}}
\def\PY#1#2{\PY@reset\PY@toks#1+\relax+\PY@do{#2}}

\expandafter\def\csname PY@tok@w\endcsname{\def\PY@tc##1{\textcolor[rgb]{0.73,0.73,0.73}{##1}}}
\expandafter\def\csname PY@tok@c\endcsname{\let\PY@it=\textit\def\PY@tc##1{\textcolor[rgb]{0.25,0.50,0.50}{##1}}}
\expandafter\def\csname PY@tok@cp\endcsname{\def\PY@tc##1{\textcolor[rgb]{0.74,0.48,0.00}{##1}}}
\expandafter\def\csname PY@tok@k\endcsname{\let\PY@bf=\textbf\def\PY@tc##1{\textcolor[rgb]{0.00,0.50,0.00}{##1}}}
\expandafter\def\csname PY@tok@kp\endcsname{\def\PY@tc##1{\textcolor[rgb]{0.00,0.50,0.00}{##1}}}
\expandafter\def\csname PY@tok@kt\endcsname{\def\PY@tc##1{\textcolor[rgb]{0.69,0.00,0.25}{##1}}}
\expandafter\def\csname PY@tok@o\endcsname{\def\PY@tc##1{\textcolor[rgb]{0.40,0.40,0.40}{##1}}}
\expandafter\def\csname PY@tok@ow\endcsname{\let\PY@bf=\textbf\def\PY@tc##1{\textcolor[rgb]{0.67,0.13,1.00}{##1}}}
\expandafter\def\csname PY@tok@nb\endcsname{\def\PY@tc##1{\textcolor[rgb]{0.00,0.50,0.00}{##1}}}
\expandafter\def\csname PY@tok@nf\endcsname{\def\PY@tc##1{\textcolor[rgb]{0.00,0.00,1.00}{##1}}}
\expandafter\def\csname PY@tok@nc\endcsname{\let\PY@bf=\textbf\def\PY@tc##1{\textcolor[rgb]{0.00,0.00,1.00}{##1}}}
\expandafter\def\csname PY@tok@nn\endcsname{\let\PY@bf=\textbf\def\PY@tc##1{\textcolor[rgb]{0.00,0.00,1.00}{##1}}}
\expandafter\def\csname PY@tok@ne\endcsname{\let\PY@bf=\textbf\def\PY@tc##1{\textcolor[rgb]{0.82,0.25,0.23}{##1}}}
\expandafter\def\csname PY@tok@nv\endcsname{\def\PY@tc##1{\textcolor[rgb]{0.10,0.09,0.49}{##1}}}
\expandafter\def\csname PY@tok@no\endcsname{\def\PY@tc##1{\textcolor[rgb]{0.53,0.00,0.00}{##1}}}
\expandafter\def\csname PY@tok@nl\endcsname{\def\PY@tc##1{\textcolor[rgb]{0.63,0.63,0.00}{##1}}}
\expandafter\def\csname PY@tok@ni\endcsname{\let\PY@bf=\textbf\def\PY@tc##1{\textcolor[rgb]{0.60,0.60,0.60}{##1}}}
\expandafter\def\csname PY@tok@na\endcsname{\def\PY@tc##1{\textcolor[rgb]{0.49,0.56,0.16}{##1}}}
\expandafter\def\csname PY@tok@nt\endcsname{\let\PY@bf=\textbf\def\PY@tc##1{\textcolor[rgb]{0.00,0.50,0.00}{##1}}}
\expandafter\def\csname PY@tok@nd\endcsname{\def\PY@tc##1{\textcolor[rgb]{0.67,0.13,1.00}{##1}}}
\expandafter\def\csname PY@tok@s\endcsname{\def\PY@tc##1{\textcolor[rgb]{0.73,0.13,0.13}{##1}}}
\expandafter\def\csname PY@tok@sd\endcsname{\let\PY@it=\textit\def\PY@tc##1{\textcolor[rgb]{0.73,0.13,0.13}{##1}}}
\expandafter\def\csname PY@tok@si\endcsname{\let\PY@bf=\textbf\def\PY@tc##1{\textcolor[rgb]{0.73,0.40,0.53}{##1}}}
\expandafter\def\csname PY@tok@se\endcsname{\let\PY@bf=\textbf\def\PY@tc##1{\textcolor[rgb]{0.73,0.40,0.13}{##1}}}
\expandafter\def\csname PY@tok@sr\endcsname{\def\PY@tc##1{\textcolor[rgb]{0.73,0.40,0.53}{##1}}}
\expandafter\def\csname PY@tok@ss\endcsname{\def\PY@tc##1{\textcolor[rgb]{0.10,0.09,0.49}{##1}}}
\expandafter\def\csname PY@tok@sx\endcsname{\def\PY@tc##1{\textcolor[rgb]{0.00,0.50,0.00}{##1}}}
\expandafter\def\csname PY@tok@m\endcsname{\def\PY@tc##1{\textcolor[rgb]{0.40,0.40,0.40}{##1}}}
\expandafter\def\csname PY@tok@gh\endcsname{\let\PY@bf=\textbf\def\PY@tc##1{\textcolor[rgb]{0.00,0.00,0.50}{##1}}}
\expandafter\def\csname PY@tok@gu\endcsname{\let\PY@bf=\textbf\def\PY@tc##1{\textcolor[rgb]{0.50,0.00,0.50}{##1}}}
\expandafter\def\csname PY@tok@gd\endcsname{\def\PY@tc##1{\textcolor[rgb]{0.63,0.00,0.00}{##1}}}
\expandafter\def\csname PY@tok@gi\endcsname{\def\PY@tc##1{\textcolor[rgb]{0.00,0.63,0.00}{##1}}}
\expandafter\def\csname PY@tok@gr\endcsname{\def\PY@tc##1{\textcolor[rgb]{1.00,0.00,0.00}{##1}}}
\expandafter\def\csname PY@tok@ge\endcsname{\let\PY@it=\textit}
\expandafter\def\csname PY@tok@gs\endcsname{\let\PY@bf=\textbf}
\expandafter\def\csname PY@tok@gp\endcsname{\let\PY@bf=\textbf\def\PY@tc##1{\textcolor[rgb]{0.00,0.00,0.50}{##1}}}
\expandafter\def\csname PY@tok@go\endcsname{\def\PY@tc##1{\textcolor[rgb]{0.53,0.53,0.53}{##1}}}
\expandafter\def\csname PY@tok@gt\endcsname{\def\PY@tc##1{\textcolor[rgb]{0.00,0.27,0.87}{##1}}}
\expandafter\def\csname PY@tok@err\endcsname{\def\PY@bc##1{\setlength{\fboxsep}{0pt}\fcolorbox[rgb]{1.00,0.00,0.00}{1,1,1}{\strut ##1}}}
\expandafter\def\csname PY@tok@kc\endcsname{\let\PY@bf=\textbf\def\PY@tc##1{\textcolor[rgb]{0.00,0.50,0.00}{##1}}}
\expandafter\def\csname PY@tok@kd\endcsname{\let\PY@bf=\textbf\def\PY@tc##1{\textcolor[rgb]{0.00,0.50,0.00}{##1}}}
\expandafter\def\csname PY@tok@kn\endcsname{\let\PY@bf=\textbf\def\PY@tc##1{\textcolor[rgb]{0.00,0.50,0.00}{##1}}}
\expandafter\def\csname PY@tok@kr\endcsname{\let\PY@bf=\textbf\def\PY@tc##1{\textcolor[rgb]{0.00,0.50,0.00}{##1}}}
\expandafter\def\csname PY@tok@bp\endcsname{\def\PY@tc##1{\textcolor[rgb]{0.00,0.50,0.00}{##1}}}
\expandafter\def\csname PY@tok@fm\endcsname{\def\PY@tc##1{\textcolor[rgb]{0.00,0.00,1.00}{##1}}}
\expandafter\def\csname PY@tok@vc\endcsname{\def\PY@tc##1{\textcolor[rgb]{0.10,0.09,0.49}{##1}}}
\expandafter\def\csname PY@tok@vg\endcsname{\def\PY@tc##1{\textcolor[rgb]{0.10,0.09,0.49}{##1}}}
\expandafter\def\csname PY@tok@vi\endcsname{\def\PY@tc##1{\textcolor[rgb]{0.10,0.09,0.49}{##1}}}
\expandafter\def\csname PY@tok@vm\endcsname{\def\PY@tc##1{\textcolor[rgb]{0.10,0.09,0.49}{##1}}}
\expandafter\def\csname PY@tok@sa\endcsname{\def\PY@tc##1{\textcolor[rgb]{0.73,0.13,0.13}{##1}}}
\expandafter\def\csname PY@tok@sb\endcsname{\def\PY@tc##1{\textcolor[rgb]{0.73,0.13,0.13}{##1}}}
\expandafter\def\csname PY@tok@sc\endcsname{\def\PY@tc##1{\textcolor[rgb]{0.73,0.13,0.13}{##1}}}
\expandafter\def\csname PY@tok@dl\endcsname{\def\PY@tc##1{\textcolor[rgb]{0.73,0.13,0.13}{##1}}}
\expandafter\def\csname PY@tok@s2\endcsname{\def\PY@tc##1{\textcolor[rgb]{0.73,0.13,0.13}{##1}}}
\expandafter\def\csname PY@tok@sh\endcsname{\def\PY@tc##1{\textcolor[rgb]{0.73,0.13,0.13}{##1}}}
\expandafter\def\csname PY@tok@s1\endcsname{\def\PY@tc##1{\textcolor[rgb]{0.73,0.13,0.13}{##1}}}
\expandafter\def\csname PY@tok@mb\endcsname{\def\PY@tc##1{\textcolor[rgb]{0.40,0.40,0.40}{##1}}}
\expandafter\def\csname PY@tok@mf\endcsname{\def\PY@tc##1{\textcolor[rgb]{0.40,0.40,0.40}{##1}}}
\expandafter\def\csname PY@tok@mh\endcsname{\def\PY@tc##1{\textcolor[rgb]{0.40,0.40,0.40}{##1}}}
\expandafter\def\csname PY@tok@mi\endcsname{\def\PY@tc##1{\textcolor[rgb]{0.40,0.40,0.40}{##1}}}
\expandafter\def\csname PY@tok@il\endcsname{\def\PY@tc##1{\textcolor[rgb]{0.40,0.40,0.40}{##1}}}
\expandafter\def\csname PY@tok@mo\endcsname{\def\PY@tc##1{\textcolor[rgb]{0.40,0.40,0.40}{##1}}}
\expandafter\def\csname PY@tok@ch\endcsname{\let\PY@it=\textit\def\PY@tc##1{\textcolor[rgb]{0.25,0.50,0.50}{##1}}}
\expandafter\def\csname PY@tok@cm\endcsname{\let\PY@it=\textit\def\PY@tc##1{\textcolor[rgb]{0.25,0.50,0.50}{##1}}}
\expandafter\def\csname PY@tok@cpf\endcsname{\let\PY@it=\textit\def\PY@tc##1{\textcolor[rgb]{0.25,0.50,0.50}{##1}}}
\expandafter\def\csname PY@tok@c1\endcsname{\let\PY@it=\textit\def\PY@tc##1{\textcolor[rgb]{0.25,0.50,0.50}{##1}}}
\expandafter\def\csname PY@tok@cs\endcsname{\let\PY@it=\textit\def\PY@tc##1{\textcolor[rgb]{0.25,0.50,0.50}{##1}}}

\def\PYZbs{\char`\\}
\def\PYZus{\char`\_}
\def\PYZob{\char`\{}
\def\PYZcb{\char`\}}
\def\PYZca{\char`\^}
\def\PYZam{\char`\&}
\def\PYZlt{\char`\<}
\def\PYZgt{\char`\>}
\def\PYZsh{\char`\#}
\def\PYZpc{\char`\%}
\def\PYZdl{\char`\$}
\def\PYZhy{\char`\-}
\def\PYZsq{\char`\'}
\def\PYZdq{\char`\"}
\def\PYZti{\char`\~}
% for compatibility with earlier versions
\def\PYZat{@}
\def\PYZlb{[}
\def\PYZrb{]}
\makeatother


    % Exact colors from NB
    \definecolor{incolor}{rgb}{0.0, 0.0, 0.5}
    \definecolor{outcolor}{rgb}{0.545, 0.0, 0.0}



    
    % Prevent overflowing lines due to hard-to-break entities
    \sloppy 
    % Setup hyperref package
    \hypersetup{
      breaklinks=true,  % so long urls are correctly broken across lines
      colorlinks=true,
      urlcolor=urlcolor,
      linkcolor=linkcolor,
      citecolor=citecolor,
      }
    % Slightly bigger margins than the latex defaults
    
    \geometry{verbose,tmargin=1in,bmargin=1in,lmargin=1in,rmargin=1in}
    
    

    \begin{document}
    
    
    \maketitle
    
    

    
    \section{Programming Exercise 1: Linear
Regression}\label{programming-exercise-1-linear-regression}

\subsection{Introduction}\label{introduction}

In this exercise, you will implement linear regression and get to see it
work on data. Before starting on this programming exercise, we strongly
recommend watching the video lectures and completing the review
questions for the associated topics.

All the information you need for solving this assignment is in this
notebook, and all the code you will be implementing will take place
within this notebook. The assignment can be promptly submitted to the
coursera grader directly from this notebook (code and instructions are
included below).

Before we begin with the exercises, we need to import all libraries
required for this programming exercise. Throughout the course, we will
be using \href{http://www.numpy.org/}{\texttt{numpy}} for all arrays and
matrix operations, and
\href{https://matplotlib.org/}{\texttt{matplotlib}} for plotting.

You can find instructions on how to install required libraries in the
README file in the
\href{https://github.com/dibgerge/ml-coursera-python-assignments}{github
repository}.

    \begin{Verbatim}[commandchars=\\\{\}]
{\color{incolor}In [{\color{incolor}7}]:} \PY{c+c1}{\PYZsh{} used for manipulating directory paths}
        \PY{k+kn}{import} \PY{n+nn}{os}
        
        \PY{c+c1}{\PYZsh{} Scientific and vector computation for python}
        \PY{k+kn}{import} \PY{n+nn}{numpy} \PY{k}{as} \PY{n+nn}{np}
        
        \PY{c+c1}{\PYZsh{} Plotting library}
        \PY{k+kn}{from} \PY{n+nn}{matplotlib} \PY{k}{import} \PY{n}{pyplot}
        \PY{k+kn}{from} \PY{n+nn}{mpl\PYZus{}toolkits}\PY{n+nn}{.}\PY{n+nn}{mplot3d} \PY{k}{import} \PY{n}{Axes3D}  \PY{c+c1}{\PYZsh{} needed to plot 3\PYZhy{}D surfaces}
        
        \PY{c+c1}{\PYZsh{} library written for this exercise providing additional functions for assignment submission, and others}
        \PY{k+kn}{import} \PY{n+nn}{utils} 
        
        \PY{c+c1}{\PYZsh{} define the submission/grader object for this exercise}
        \PY{n}{grader} \PY{o}{=} \PY{n}{utils}\PY{o}{.}\PY{n}{Grader}\PY{p}{(}\PY{p}{)}
        
        \PY{c+c1}{\PYZsh{} tells matplotlib to embed plots within the notebook}
        \PY{o}{\PYZpc{}}\PY{k}{matplotlib} inline
\end{Verbatim}

    \subsection{Submission and Grading}\label{submission-and-grading}

After completing each part of the assignment, be sure to submit your
solutions to the grader.

For this programming exercise, you are only required to complete the
first part of the exercise to implement linear regression with one
variable. The second part of the exercise, which is optional, covers
linear regression with multiple variables. The following is a breakdown
of how each part of this exercise is scored.

\textbf{Required Exercises}

\begin{longtable}[]{@{}lllc@{}}
\toprule
Section & Part & Submitted Function & Points\tabularnewline
\midrule
\endhead
1 & Section \ref{section1} & Section \ref{warmupexercise} &
10\tabularnewline
2 & Section \ref{section2} & Section \ref{computecost} &
40\tabularnewline
3 & Section \ref{section3} & Section \ref{gradientdescent} &
50\tabularnewline
& Total Points & & 100\tabularnewline
\bottomrule
\end{longtable}

\textbf{Optional Exercises}

\begin{longtable}[]{@{}clcc@{}}
\toprule
Section & Part & Submitted Function & Points\tabularnewline
\midrule
\endhead
4 & Section \ref{section4} & Section \ref{featurenormalize} &
0\tabularnewline
5 & Section \ref{section5} & Section \ref{computecostmulti} &
0\tabularnewline
6 & Section \ref{section5} & Section \ref{gradientdescentmulti} &
0\tabularnewline
7 & Section \ref{section7} & Section \ref{normaleqn} & 0\tabularnewline
\bottomrule
\end{longtable}

You are allowed to submit your solutions multiple times, and we will
take only the highest score into consideration.

At the end of each section in this notebook, we have a cell which
contains code for submitting the solutions thus far to the grader.
Execute the cell to see your score up to the current section. For all
your work to be submitted properly, you must execute those cells at
least once. They must also be re-executed everytime the submitted
function is updated.

\subsection{Debugging}\label{debugging}

Here are some things to keep in mind throughout this exercise:

\begin{itemize}
\item
  Python array indices start from zero, not one (contrary to
  OCTAVE/MATLAB).
\item
  There is an important distinction between python arrays (called
  \texttt{list} or \texttt{tuple}) and \texttt{numpy} arrays. You should
  use \texttt{numpy} arrays in all your computations. Vector/matrix
  operations work only with \texttt{numpy} arrays. Python lists do not
  support vector operations (you need to use for loops).
\item
  If you are seeing many errors at runtime, inspect your matrix
  operations to make sure that you are adding and multiplying matrices
  of compatible dimensions. Printing the dimensions of \texttt{numpy}
  arrays using the \texttt{shape} property will help you debug.
\item
  By default, \texttt{numpy} interprets math operators to be
  element-wise operators. If you want to do matrix multiplication, you
  need to use the \texttt{dot} function in \texttt{numpy}. For, example
  if \texttt{A} and \texttt{B} are two \texttt{numpy} matrices, then the
  matrix operation AB is \texttt{np.dot(A,\ B)}. Note that for
  2-dimensional matrices or vectors (1-dimensional), this is also
  equivalent to \texttt{A@B} (requires python \textgreater{}= 3.5).
\end{itemize}

     \#\# 1 Simple python and \texttt{numpy} function

The first part of this assignment gives you practice with python and
\texttt{numpy} syntax and the homework submission process. In the next
cell, you will find the outline of a \texttt{python} function. Modify it
to return a 5 x 5 identity matrix by filling in the following code:

\begin{Shaded}
\begin{Highlighting}[]
\NormalTok{A }\OperatorTok{=}\NormalTok{ np.eye(}\DecValTok{5}\NormalTok{)}
\end{Highlighting}
\end{Shaded}

    \begin{Verbatim}[commandchars=\\\{\}]
{\color{incolor}In [{\color{incolor}8}]:} \PY{k}{def} \PY{n+nf}{warmUpExercise}\PY{p}{(}\PY{p}{)}\PY{p}{:}
            \PY{l+s+sd}{\PYZdq{}\PYZdq{}\PYZdq{}}
        \PY{l+s+sd}{    Example function in Python which computes the identity matrix.}
        \PY{l+s+sd}{    }
        \PY{l+s+sd}{    Returns}
        \PY{l+s+sd}{    \PYZhy{}\PYZhy{}\PYZhy{}\PYZhy{}\PYZhy{}\PYZhy{}\PYZhy{}}
        \PY{l+s+sd}{    A : array\PYZus{}like}
        \PY{l+s+sd}{        The 5x5 identity matrix.}
        \PY{l+s+sd}{    }
        \PY{l+s+sd}{    Instructions}
        \PY{l+s+sd}{    \PYZhy{}\PYZhy{}\PYZhy{}\PYZhy{}\PYZhy{}\PYZhy{}\PYZhy{}\PYZhy{}\PYZhy{}\PYZhy{}\PYZhy{}\PYZhy{}}
        \PY{l+s+sd}{    Return the 5x5 identity matrix.}
        \PY{l+s+sd}{    \PYZdq{}\PYZdq{}\PYZdq{}}    
            \PY{c+c1}{\PYZsh{} ======== YOUR CODE HERE ======}
            \PY{n}{A} \PY{o}{=} \PY{n}{np}\PY{o}{.}\PY{n}{eye}\PY{p}{(}\PY{l+m+mi}{5}\PY{p}{)}   \PY{c+c1}{\PYZsh{} modify this line}
            
            \PY{c+c1}{\PYZsh{} ==============================}
            \PY{k}{return} \PY{n}{A}
\end{Verbatim}

    The previous cell only defines the function \texttt{warmUpExercise}. We
can now run it by executing the following cell to see its output. You
should see output similar to the following:

\begin{Shaded}
\begin{Highlighting}[]
\NormalTok{array([[ }\DecValTok{1}\NormalTok{.,  }\DecValTok{0}\NormalTok{.,  }\DecValTok{0}\NormalTok{.,  }\DecValTok{0}\NormalTok{.,  }\DecValTok{0}\NormalTok{.],}
\NormalTok{       [ }\DecValTok{0}\NormalTok{.,  }\DecValTok{1}\NormalTok{.,  }\DecValTok{0}\NormalTok{.,  }\DecValTok{0}\NormalTok{.,  }\DecValTok{0}\NormalTok{.],}
\NormalTok{       [ }\DecValTok{0}\NormalTok{.,  }\DecValTok{0}\NormalTok{.,  }\DecValTok{1}\NormalTok{.,  }\DecValTok{0}\NormalTok{.,  }\DecValTok{0}\NormalTok{.],}
\NormalTok{       [ }\DecValTok{0}\NormalTok{.,  }\DecValTok{0}\NormalTok{.,  }\DecValTok{0}\NormalTok{.,  }\DecValTok{1}\NormalTok{.,  }\DecValTok{0}\NormalTok{.],}
\NormalTok{       [ }\DecValTok{0}\NormalTok{.,  }\DecValTok{0}\NormalTok{.,  }\DecValTok{0}\NormalTok{.,  }\DecValTok{0}\NormalTok{.,  }\DecValTok{1}\NormalTok{.]])}
\end{Highlighting}
\end{Shaded}

    \begin{Verbatim}[commandchars=\\\{\}]
{\color{incolor}In [{\color{incolor}9}]:} \PY{n}{warmUpExercise}\PY{p}{(}\PY{p}{)}
\end{Verbatim}

\begin{Verbatim}[commandchars=\\\{\}]
{\color{outcolor}Out[{\color{outcolor}9}]:} array([[1., 0., 0., 0., 0.],
               [0., 1., 0., 0., 0.],
               [0., 0., 1., 0., 0.],
               [0., 0., 0., 1., 0.],
               [0., 0., 0., 0., 1.]])
\end{Verbatim}
            
    \subsubsection{1.1 Submitting solutions}\label{submitting-solutions}

After completing a part of the exercise, you can submit your solutions
for grading by first adding the function you modified to the grader
object, and then sending your function to Coursera for grading.

The grader will prompt you for your login e-mail and submission token.
You can obtain a submission token from the web page for the assignment.
You are allowed to submit your solutions multiple times, and we will
take only the highest score into consideration.

Execute the next cell to grade your solution to the first part of this
exercise.

\emph{You should now submit your solutions.}

    \begin{Verbatim}[commandchars=\\\{\}]
{\color{incolor}In [{\color{incolor}10}]:} \PY{c+c1}{\PYZsh{} appends the implemented function in part 1 to the grader object}
         \PY{n}{grader}\PY{p}{[}\PY{l+m+mi}{1}\PY{p}{]} \PY{o}{=} \PY{n}{warmUpExercise}
         
         \PY{c+c1}{\PYZsh{} send the added functions to coursera grader for getting a grade on this part}
         \PY{n}{grader}\PY{o}{.}\PY{n}{grade}\PY{p}{(}\PY{p}{)}
\end{Verbatim}

    \begin{Verbatim}[commandchars=\\\{\}]

Submitting Solutions | Programming Exercise linear-regression


    \end{Verbatim}

    \begin{Verbatim}[commandchars=\\\{\}]
Use token from last successful submission (miken024@live.com)? (Y/n):  y

    \end{Verbatim}

    \begin{Verbatim}[commandchars=\\\{\}]
                                  Part Name |     Score | Feedback
                                  --------- |     ----- | --------
                           Warm up exercise |  10 /  10 | Nice work!
          Computing Cost (for one variable) |   0 /  40 | 
        Gradient Descent (for one variable) |   0 /  50 | 
                      Feature Normalization |   0 /   0 | 
    Computing Cost (for multiple variables) |   0 /   0 | 
  Gradient Descent (for multiple variables) |   0 /   0 | 
                           Normal Equations |   0 /   0 | 
                                  --------------------------------
                                            |  10 / 100 |  


    \end{Verbatim}

    \subsection{2 Linear regression with one
variable}\label{linear-regression-with-one-variable}

Now you will implement linear regression with one variable to predict
profits for a food truck. Suppose you are the CEO of a restaurant
franchise and are considering different cities for opening a new outlet.
The chain already has trucks in various cities and you have data for
profits and populations from the cities. You would like to use this data
to help you select which city to expand to next.

The file \texttt{Data/ex1data1.txt} contains the dataset for our linear
regression problem. The first column is the population of a city (in
10,000s) and the second column is the profit of a food truck in that
city (in \$10,000s). A negative value for profit indicates a loss.

We provide you with the code needed to load this data. The dataset is
loaded from the data file into the variables \texttt{x} and \texttt{y}:

    \begin{Verbatim}[commandchars=\\\{\}]
{\color{incolor}In [{\color{incolor}11}]:} \PY{c+c1}{\PYZsh{} Read comma separated data}
         \PY{n}{data} \PY{o}{=} \PY{n}{np}\PY{o}{.}\PY{n}{loadtxt}\PY{p}{(}\PY{n}{os}\PY{o}{.}\PY{n}{path}\PY{o}{.}\PY{n}{join}\PY{p}{(}\PY{l+s+s1}{\PYZsq{}}\PY{l+s+s1}{Data}\PY{l+s+s1}{\PYZsq{}}\PY{p}{,} \PY{l+s+s1}{\PYZsq{}}\PY{l+s+s1}{ex1data1.txt}\PY{l+s+s1}{\PYZsq{}}\PY{p}{)}\PY{p}{,} \PY{n}{delimiter}\PY{o}{=}\PY{l+s+s1}{\PYZsq{}}\PY{l+s+s1}{,}\PY{l+s+s1}{\PYZsq{}}\PY{p}{)}
         \PY{n}{X}\PY{p}{,} \PY{n}{y} \PY{o}{=} \PY{n}{data}\PY{p}{[}\PY{p}{:}\PY{p}{,} \PY{l+m+mi}{0}\PY{p}{]}\PY{p}{,} \PY{n}{data}\PY{p}{[}\PY{p}{:}\PY{p}{,} \PY{l+m+mi}{1}\PY{p}{]}
         
         \PY{n}{m} \PY{o}{=} \PY{n}{y}\PY{o}{.}\PY{n}{size}  \PY{c+c1}{\PYZsh{} number of training examples}
\end{Verbatim}

    \subsubsection{2.1 Plotting the Data}\label{plotting-the-data}

Before starting on any task, it is often useful to understand the data
by visualizing it. For this dataset, you can use a scatter plot to
visualize the data, since it has only two properties to plot (profit and
population). Many other problems that you will encounter in real life
are multi-dimensional and cannot be plotted on a 2-d plot. There are
many plotting libraries in python (see this
\href{https://blog.modeanalytics.com/python-data-visualization-libraries/}{blog
post} for a good summary of the most popular ones).

In this course, we will be exclusively using \texttt{matplotlib} to do
all our plotting. \texttt{matplotlib} is one of the most popular
scientific plotting libraries in python and has extensive tools and
functions to make beautiful plots. \texttt{pyplot} is a module within
\texttt{matplotlib} which provides a simplified interface to
\texttt{matplotlib}'s most common plotting tasks, mimicking MATLAB's
plotting interface.

You might have noticed that we have imported the \texttt{pyplot} module
at the beginning of this exercise using the command
\texttt{from\ matplotlib\ import\ pyplot}. This is rather uncommon, and
if you look at python code elsewhere or in the \texttt{matplotlib}
tutorials, you will see that the module is named \texttt{plt}. This is
used by module renaming by using the import command
\texttt{import\ matplotlib.pyplot\ as\ plt}. We will not using the short
name of \texttt{pyplot} module in this class exercises, but you should
be aware of this deviation from norm.

In the following part, your first job is to complete the
\texttt{plotData} function below. Modify the function and fill in the
following code:

\begin{Shaded}
\begin{Highlighting}[]
\NormalTok{    pyplot.plot(x, y, }\StringTok{'ro'}\NormalTok{, ms}\OperatorTok{=}\DecValTok{10}\NormalTok{, mec}\OperatorTok{=}\StringTok{'k'}\NormalTok{)}
\NormalTok{    pyplot.ylabel(}\StringTok{'Profit in $10,000'}\NormalTok{)}
\NormalTok{    pyplot.xlabel(}\StringTok{'Population of City in 10,000s'}\NormalTok{)}
\end{Highlighting}
\end{Shaded}

    \begin{Verbatim}[commandchars=\\\{\}]
{\color{incolor}In [{\color{incolor}12}]:} \PY{k}{def} \PY{n+nf}{plotData}\PY{p}{(}\PY{n}{x}\PY{p}{,} \PY{n}{y}\PY{p}{)}\PY{p}{:}
             \PY{l+s+sd}{\PYZdq{}\PYZdq{}\PYZdq{}}
         \PY{l+s+sd}{    Plots the data points x and y into a new figure. Plots the data }
         \PY{l+s+sd}{    points and gives the figure axes labels of population and profit.}
         \PY{l+s+sd}{    }
         \PY{l+s+sd}{    Parameters}
         \PY{l+s+sd}{    \PYZhy{}\PYZhy{}\PYZhy{}\PYZhy{}\PYZhy{}\PYZhy{}\PYZhy{}\PYZhy{}\PYZhy{}\PYZhy{}}
         \PY{l+s+sd}{    x : array\PYZus{}like}
         \PY{l+s+sd}{        Data point values for x\PYZhy{}axis.}
         
         \PY{l+s+sd}{    y : array\PYZus{}like}
         \PY{l+s+sd}{        Data point values for y\PYZhy{}axis. Note x and y should have the same size.}
         \PY{l+s+sd}{    }
         \PY{l+s+sd}{    Instructions}
         \PY{l+s+sd}{    \PYZhy{}\PYZhy{}\PYZhy{}\PYZhy{}\PYZhy{}\PYZhy{}\PYZhy{}\PYZhy{}\PYZhy{}\PYZhy{}\PYZhy{}\PYZhy{}}
         \PY{l+s+sd}{    Plot the training data into a figure using the \PYZdq{}figure\PYZdq{} and \PYZdq{}plot\PYZdq{}}
         \PY{l+s+sd}{    functions. Set the axes labels using the \PYZdq{}xlabel\PYZdq{} and \PYZdq{}ylabel\PYZdq{} functions.}
         \PY{l+s+sd}{    Assume the population and revenue data have been passed in as the x}
         \PY{l+s+sd}{    and y arguments of this function.    }
         \PY{l+s+sd}{    }
         \PY{l+s+sd}{    Hint}
         \PY{l+s+sd}{    \PYZhy{}\PYZhy{}\PYZhy{}\PYZhy{}}
         \PY{l+s+sd}{    You can use the \PYZsq{}ro\PYZsq{} option with plot to have the markers}
         \PY{l+s+sd}{    appear as red circles. Furthermore, you can make the markers larger by}
         \PY{l+s+sd}{    using plot(..., \PYZsq{}ro\PYZsq{}, ms=10), where `ms` refers to marker size. You }
         \PY{l+s+sd}{    can also set the marker edge color using the `mec` property.}
         \PY{l+s+sd}{    \PYZdq{}\PYZdq{}\PYZdq{}}
             \PY{n}{fig} \PY{o}{=} \PY{n}{pyplot}\PY{o}{.}\PY{n}{figure}\PY{p}{(}\PY{p}{)}  \PY{c+c1}{\PYZsh{} open a new figure}
             
             \PY{c+c1}{\PYZsh{} ====================== YOUR CODE HERE ======================= }
             \PY{n}{pyplot}\PY{o}{.}\PY{n}{plot}\PY{p}{(}\PY{n}{x}\PY{p}{,} \PY{n}{y}\PY{p}{,} \PY{l+s+s1}{\PYZsq{}}\PY{l+s+s1}{ro}\PY{l+s+s1}{\PYZsq{}}\PY{p}{,} \PY{n}{ms}\PY{o}{=}\PY{l+m+mi}{10}\PY{p}{,} \PY{n}{mec} \PY{o}{=} \PY{l+s+s1}{\PYZsq{}}\PY{l+s+s1}{k}\PY{l+s+s1}{\PYZsq{}}\PY{p}{)}
             \PY{n}{pyplot}\PY{o}{.}\PY{n}{ylabel}\PY{p}{(}\PY{l+s+s1}{\PYZsq{}}\PY{l+s+s1}{Profit (\PYZdl{}10,000)}\PY{l+s+s1}{\PYZsq{}}\PY{p}{)}
             \PY{n}{pyplot}\PY{o}{.}\PY{n}{xlabel}\PY{p}{(}\PY{l+s+s1}{\PYZsq{}}\PY{l+s+s1}{Population of City (10,000)}\PY{l+s+s1}{\PYZsq{}}\PY{p}{)}
         
             \PY{c+c1}{\PYZsh{} =============================================================}
\end{Verbatim}

    Now run the defined function with the loaded data to visualize the data.
The end result should look like the following figure:

\begin{figure}
\centering
\includegraphics{Figures/dataset1.png}
\caption{}
\end{figure}

Execute the next cell to visualize the data.

    \begin{Verbatim}[commandchars=\\\{\}]
{\color{incolor}In [{\color{incolor}13}]:} \PY{n}{plotData}\PY{p}{(}\PY{n}{X}\PY{p}{,} \PY{n}{y}\PY{p}{)}
\end{Verbatim}

    \begin{center}
    \adjustimage{max size={0.9\linewidth}{0.9\paperheight}}{output_14_0.png}
    \end{center}
    { \hspace*{\fill} \\}
    
    To quickly learn more about the \texttt{matplotlib} plot function and
what arguments you can provide to it, you can type \texttt{?pyplot.plot}
in a cell within the jupyter notebook. This opens a separate page
showing the documentation for the requested function. You can also
search online for plotting documentation.

To set the markers to red circles, we used the option
\texttt{\textquotesingle{}or\textquotesingle{}} within the \texttt{plot}
function.

    \begin{Verbatim}[commandchars=\\\{\}]
{\color{incolor}In [{\color{incolor}14}]:} \PY{o}{?}pyplot.plot
\end{Verbatim}

    
    \begin{verbatim}
Signature: pyplot.plot(*args, scalex=True, scaley=True, data=None, **kwargs)
Docstring:
Plot y versus x as lines and/or markers.

Call signatures::

    plot([x], y, [fmt], data=None, **kwargs)
    plot([x], y, [fmt], [x2], y2, [fmt2], ..., **kwargs)

The coordinates of the points or line nodes are given by *x*, *y*.

The optional parameter *fmt* is a convenient way for defining basic
formatting like color, marker and linestyle. It's a shortcut string
notation described in the *Notes* section below.

>>> plot(x, y)        # plot x and y using default line style and color
>>> plot(x, y, 'bo')  # plot x and y using blue circle markers
>>> plot(y)           # plot y using x as index array 0..N-1
>>> plot(y, 'r+')     # ditto, but with red plusses

You can use `.Line2D` properties as keyword arguments for more
control on the appearance. Line properties and *fmt* can be mixed.
The following two calls yield identical results:

>>> plot(x, y, 'go--', linewidth=2, markersize=12)
>>> plot(x, y, color='green', marker='o', linestyle='dashed',
...      linewidth=2, markersize=12)

When conflicting with *fmt*, keyword arguments take precedence.

**Plotting labelled data**

There's a convenient way for plotting objects with labelled data (i.e.
data that can be accessed by index ``obj['y']``). Instead of giving
the data in *x* and *y*, you can provide the object in the *data*
parameter and just give the labels for *x* and *y*::

>>> plot('xlabel', 'ylabel', data=obj)

All indexable objects are supported. This could e.g. be a `dict`, a
`pandas.DataFame` or a structured numpy array.


**Plotting multiple sets of data**

There are various ways to plot multiple sets of data.

- The most straight forward way is just to call `plot` multiple times.
  Example:

  >>> plot(x1, y1, 'bo')
  >>> plot(x2, y2, 'go')

- Alternatively, if your data is already a 2d array, you can pass it
  directly to *x*, *y*. A separate data set will be drawn for every
  column.

  Example: an array ``a`` where the first column represents the *x*
  values and the other columns are the *y* columns::

  >>> plot(a[0], a[1:])

- The third way is to specify multiple sets of *[x]*, *y*, *[fmt]*
  groups::

  >>> plot(x1, y1, 'g^', x2, y2, 'g-')

  In this case, any additional keyword argument applies to all
  datasets. Also this syntax cannot be combined with the *data*
  parameter.

By default, each line is assigned a different style specified by a
'style cycle'. The *fmt* and line property parameters are only
necessary if you want explicit deviations from these defaults.
Alternatively, you can also change the style cycle using the
'axes.prop_cycle' rcParam.

Parameters
----------
x, y : array-like or scalar
    The horizontal / vertical coordinates of the data points.
    *x* values are optional. If not given, they default to
    ``[0, ..., N-1]``.

    Commonly, these parameters are arrays of length N. However,
    scalars are supported as well (equivalent to an array with
    constant value).

    The parameters can also be 2-dimensional. Then, the columns
    represent separate data sets.

fmt : str, optional
    A format string, e.g. 'ro' for red circles. See the *Notes*
    section for a full description of the format strings.

    Format strings are just an abbreviation for quickly setting
    basic line properties. All of these and more can also be
    controlled by keyword arguments.

data : indexable object, optional
    An object with labelled data. If given, provide the label names to
    plot in *x* and *y*.

    .. note::
        Technically there's a slight ambiguity in calls where the
        second label is a valid *fmt*. `plot('n', 'o', data=obj)`
        could be `plt(x, y)` or `plt(y, fmt)`. In such cases,
        the former interpretation is chosen, but a warning is issued.
        You may suppress the warning by adding an empty format string
        `plot('n', 'o', '', data=obj)`.


Other Parameters
----------------
scalex, scaley : bool, optional, default: True
    These parameters determined if the view limits are adapted to
    the data limits. The values are passed on to `autoscale_view`.

**kwargs : `.Line2D` properties, optional
    *kwargs* are used to specify properties like a line label (for
    auto legends), linewidth, antialiasing, marker face color.
    Example::

    >>> plot([1,2,3], [1,2,3], 'go-', label='line 1', linewidth=2)
    >>> plot([1,2,3], [1,4,9], 'rs',  label='line 2')

    If you make multiple lines with one plot command, the kwargs
    apply to all those lines.

    Here is a list of available `.Line2D` properties:

      agg_filter: a filter function, which takes a (m, n, 3) float array and a dpi value, and returns a (m, n, 3) array 
  alpha: float
  animated: bool
  antialiased: bool
  clip_box: `.Bbox`
  clip_on: bool
  clip_path: [(`~matplotlib.path.Path`, `.Transform`) | `.Patch` | None] 
  color: color
  contains: callable
  dash_capstyle: {'butt', 'round', 'projecting'}
  dash_joinstyle: {'miter', 'round', 'bevel'}
  dashes: sequence of floats (on/off ink in points) or (None, None)
  drawstyle: {'default', 'steps', 'steps-pre', 'steps-mid', 'steps-post'}
  figure: `.Figure`
  fillstyle: {'full', 'left', 'right', 'bottom', 'top', 'none'}
  gid: str
  in_layout: bool
  label: object
  linestyle: {'-', '--', '-.', ':', '', (offset, on-off-seq), ...}
  linewidth: float
  marker: unknown
  markeredgecolor: color
  markeredgewidth: float
  markerfacecolor: color
  markerfacecoloralt: color
  markersize: float
  markevery: unknown
  path_effects: `.AbstractPathEffect`
  picker: float or callable[[Artist, Event], Tuple[bool, dict]]
  pickradius: float
  rasterized: bool or None
  sketch_params: (scale: float, length: float, randomness: float) 
  snap: bool or None
  solid_capstyle: {'butt', 'round', 'projecting'}
  solid_joinstyle: {'miter', 'round', 'bevel'}
  transform: matplotlib.transforms.Transform
  url: str
  visible: bool
  xdata: 1D array
  ydata: 1D array
  zorder: float

Returns
-------
lines
    A list of `.Line2D` objects representing the plotted data.


See Also
--------
scatter : XY scatter plot with markers of varying size and/or color (
    sometimes also called bubble chart).


Notes
-----
**Format Strings**

A format string consists of a part for color, marker and line::

    fmt = '[color][marker][line]'

Each of them is optional. If not provided, the value from the style
cycle is used. Exception: If ``line`` is given, but no ``marker``,
the data will be a line without markers.

**Colors**

The following color abbreviations are supported:

=============    ===============================
character        color
=============    ===============================
``'b'``          blue
``'g'``          green
``'r'``          red
``'c'``          cyan
``'m'``          magenta
``'y'``          yellow
``'k'``          black
``'w'``          white
=============    ===============================

If the color is the only part of the format string, you can
additionally use any  `matplotlib.colors` spec, e.g. full names
(``'green'``) or hex strings (``'#008000'``).

**Markers**

=============    ===============================
character        description
=============    ===============================
``'.'``          point marker
``','``          pixel marker
``'o'``          circle marker
``'v'``          triangle_down marker
``'^'``          triangle_up marker
``'<'``          triangle_left marker
``'>'``          triangle_right marker
``'1'``          tri_down marker
``'2'``          tri_up marker
``'3'``          tri_left marker
``'4'``          tri_right marker
``'s'``          square marker
``'p'``          pentagon marker
``'*'``          star marker
``'h'``          hexagon1 marker
``'H'``          hexagon2 marker
``'+'``          plus marker
``'x'``          x marker
``'D'``          diamond marker
``'d'``          thin_diamond marker
``'|'``          vline marker
``'_'``          hline marker
=============    ===============================

**Line Styles**

=============    ===============================
character        description
=============    ===============================
``'-'``          solid line style
``'--'``         dashed line style
``'-.'``         dash-dot line style
``':'``          dotted line style
=============    ===============================

Example format strings::

    'b'    # blue markers with default shape
    'ro'   # red circles
    'g-'   # green solid line
    '--'   # dashed line with default color
    'k^:'  # black triangle_up markers connected by a dotted line

.. note::
    In addition to the above described arguments, this function can take a
    **data** keyword argument. If such a **data** argument is given, the
    following arguments are replaced by **data[<arg>]**:

    * All arguments with the following names: 'x', 'y'.

    Objects passed as **data** must support item access (``data[<arg>]``) and
    membership test (``<arg> in data``).
File:      c:\users\mike\anaconda3\lib\site-packages\matplotlib\pyplot.py
Type:      function

    \end{verbatim}

    
     \#\#\# 2.2 Gradient Descent

In this part, you will fit the linear regression parameters \(\theta\)
to our dataset using gradient descent.

\paragraph{2.2.1 Update Equations}\label{update-equations}

The objective of linear regression is to minimize the cost function

\[ J(\theta) = \frac{1}{2m} \sum_{i=1}^m \left( h_{\theta}(x^{(i)}) - y^{(i)}\right)^2\]

where the hypothesis \(h_\theta(x)\) is given by the linear model
\[ h_\theta(x) = \theta^Tx = \theta_0 + \theta_1 x_1\]

Recall that the parameters of your model are the \(\theta_j\) values.
These are the values you will adjust to minimize cost \(J(\theta)\). One
way to do this is to use the batch gradient descent algorithm. In batch
gradient descent, each iteration performs the update

\[ \theta_j = \theta_j - \alpha \frac{1}{m} \sum_{i=1}^m \left( h_\theta(x^{(i)}) - y^{(i)}\right)x_j^{(i)} \qquad \text{simultaneously update } \theta_j \text{ for all } j\]

With each step of gradient descent, your parameters \(\theta_j\) come
closer to the optimal values that will achieve the lowest cost
J(\(\theta\)).

\textbf{Implementation Note:} We store each example as a row in the the
\(X\) matrix in Python \texttt{numpy}. To take into account the
intercept term (\(\theta_0\)), we add an additional first column to
\(X\) and set it to all ones. This allows us to treat \(\theta_0\) as
simply another 'feature'.

\paragraph{2.2.2 Implementation}\label{implementation}

We have already set up the data for linear regression. In the following
cell, we add another dimension to our data to accommodate the
\(\theta_0\) intercept term. Do NOT execute this cell more than once.

    \begin{Verbatim}[commandchars=\\\{\}]
{\color{incolor}In [{\color{incolor}15}]:} \PY{c+c1}{\PYZsh{} Add a column of ones to X. The numpy function stack joins arrays along a given axis. }
         \PY{c+c1}{\PYZsh{} The first axis (axis=0) refers to rows (training examples) }
         \PY{c+c1}{\PYZsh{} and second axis (axis=1) refers to columns (features).}
         \PY{n}{X} \PY{o}{=} \PY{n}{np}\PY{o}{.}\PY{n}{stack}\PY{p}{(}\PY{p}{[}\PY{n}{np}\PY{o}{.}\PY{n}{ones}\PY{p}{(}\PY{n}{m}\PY{p}{)}\PY{p}{,} \PY{n}{X}\PY{p}{]}\PY{p}{,} \PY{n}{axis}\PY{o}{=}\PY{l+m+mi}{1}\PY{p}{)}
\end{Verbatim}

     \#\#\#\# 2.2.3 Computing the cost \(J(\theta)\)

As you perform gradient descent to learn minimize the cost function
\(J(\theta)\), it is helpful to monitor the convergence by computing the
cost. In this section, you will implement a function to calculate
\(J(\theta)\) so you can check the convergence of your gradient descent
implementation.

Your next task is to complete the code for the function
\texttt{computeCost} which computes \(J(\theta)\). As you are doing
this, remember that the variables \(X\) and \(y\) are not scalar values.
\(X\) is a matrix whose rows represent the examples from the training
set and \(y\) is a vector whose each elemennt represent the value at a
given row of \(X\). 

    \begin{Verbatim}[commandchars=\\\{\}]
{\color{incolor}In [{\color{incolor}16}]:} \PY{k}{def} \PY{n+nf}{computeCost}\PY{p}{(}\PY{n}{X}\PY{p}{,} \PY{n}{y}\PY{p}{,} \PY{n}{theta}\PY{p}{)}\PY{p}{:}
             \PY{l+s+sd}{\PYZdq{}\PYZdq{}\PYZdq{}}
         \PY{l+s+sd}{    Compute cost for linear regression. Computes the cost of using theta as the}
         \PY{l+s+sd}{    parameter for linear regression to fit the data points in X and y.}
         \PY{l+s+sd}{    }
         \PY{l+s+sd}{    Parameters}
         \PY{l+s+sd}{    \PYZhy{}\PYZhy{}\PYZhy{}\PYZhy{}\PYZhy{}\PYZhy{}\PYZhy{}\PYZhy{}\PYZhy{}\PYZhy{}}
         \PY{l+s+sd}{    X : array\PYZus{}like}
         \PY{l+s+sd}{        The input dataset of shape (m x n+1), where m is the number of examples,}
         \PY{l+s+sd}{        and n is the number of features. We assume a vector of one\PYZsq{}s already }
         \PY{l+s+sd}{        appended to the features so we have n+1 columns.}
         \PY{l+s+sd}{    }
         \PY{l+s+sd}{    y : array\PYZus{}like}
         \PY{l+s+sd}{        The values of the function at each data point. This is a vector of}
         \PY{l+s+sd}{        shape (m, ).}
         \PY{l+s+sd}{    }
         \PY{l+s+sd}{    theta : array\PYZus{}like}
         \PY{l+s+sd}{        The parameters for the regression function. This is a vector of }
         \PY{l+s+sd}{        shape (n+1, ).}
         \PY{l+s+sd}{    }
         \PY{l+s+sd}{    Returns}
         \PY{l+s+sd}{    \PYZhy{}\PYZhy{}\PYZhy{}\PYZhy{}\PYZhy{}\PYZhy{}\PYZhy{}}
         \PY{l+s+sd}{    J : float}
         \PY{l+s+sd}{        The value of the regression cost function.}
         \PY{l+s+sd}{    }
         \PY{l+s+sd}{    Instructions}
         \PY{l+s+sd}{    \PYZhy{}\PYZhy{}\PYZhy{}\PYZhy{}\PYZhy{}\PYZhy{}\PYZhy{}\PYZhy{}\PYZhy{}\PYZhy{}\PYZhy{}\PYZhy{}}
         \PY{l+s+sd}{    Compute the cost of a particular choice of theta. }
         \PY{l+s+sd}{    You should set J to the cost.}
         \PY{l+s+sd}{    \PYZdq{}\PYZdq{}\PYZdq{}}
             
             \PY{c+c1}{\PYZsh{} initialize some useful values}
             \PY{n}{m} \PY{o}{=} \PY{n}{y}\PY{o}{.}\PY{n}{size}  \PY{c+c1}{\PYZsh{} number of training examples}
             
             \PY{c+c1}{\PYZsh{} You need to return the following variables correctly}
             \PY{n}{J} \PY{o}{=} \PY{l+m+mi}{0}
             
             \PY{c+c1}{\PYZsh{} ====================== YOUR CODE HERE =====================}
             \PY{n}{num} \PY{o}{=} \PY{n}{np}\PY{o}{.}\PY{n}{power}\PY{p}{(}\PY{p}{(}\PY{p}{(}\PY{n}{X}\PY{o}{.}\PY{n}{dot}\PY{p}{(}\PY{n}{theta}\PY{p}{)}\PY{p}{)} \PY{o}{\PYZhy{}} \PY{n}{y}\PY{p}{)}\PY{p}{,} \PY{l+m+mi}{2}\PY{p}{)}
             \PY{n}{den} \PY{o}{=} \PY{n}{np}\PY{o}{.}\PY{n}{sum}\PY{p}{(}\PY{l+m+mi}{2} \PY{o}{*} \PY{n+nb}{len}\PY{p}{(}\PY{n}{X}\PY{p}{)}\PY{p}{)}
             \PY{n}{J} \PY{o}{=} \PY{n}{np}\PY{o}{.}\PY{n}{sum}\PY{p}{(}\PY{n}{num} \PY{o}{/} \PY{n}{den}\PY{p}{)}
             
             \PY{c+c1}{\PYZsh{} ===========================================================}
             \PY{k}{return} \PY{n}{J}
\end{Verbatim}

    Once you have completed the function, the next step will run
\texttt{computeCost} two times using two different initializations of
\(\theta\). You will see the cost printed to the screen.

    \begin{Verbatim}[commandchars=\\\{\}]
{\color{incolor}In [{\color{incolor}17}]:} \PY{n}{J} \PY{o}{=} \PY{n}{computeCost}\PY{p}{(}\PY{n}{X}\PY{p}{,} \PY{n}{y}\PY{p}{,} \PY{n}{theta}\PY{o}{=}\PY{n}{np}\PY{o}{.}\PY{n}{array}\PY{p}{(}\PY{p}{[}\PY{l+m+mf}{0.0}\PY{p}{,} \PY{l+m+mf}{0.0}\PY{p}{]}\PY{p}{)}\PY{p}{)}
         \PY{n+nb}{print}\PY{p}{(}\PY{l+s+s1}{\PYZsq{}}\PY{l+s+s1}{With theta = [0, 0] }\PY{l+s+se}{\PYZbs{}n}\PY{l+s+s1}{Cost computed = }\PY{l+s+si}{\PYZpc{}.2f}\PY{l+s+s1}{\PYZsq{}} \PY{o}{\PYZpc{}} \PY{n}{J}\PY{p}{)}
         \PY{n+nb}{print}\PY{p}{(}\PY{l+s+s1}{\PYZsq{}}\PY{l+s+s1}{Expected cost value (approximately) 32.07}\PY{l+s+se}{\PYZbs{}n}\PY{l+s+s1}{\PYZsq{}}\PY{p}{)}
         
         \PY{c+c1}{\PYZsh{} further testing of the cost function}
         \PY{n}{J} \PY{o}{=} \PY{n}{computeCost}\PY{p}{(}\PY{n}{X}\PY{p}{,} \PY{n}{y}\PY{p}{,} \PY{n}{theta}\PY{o}{=}\PY{n}{np}\PY{o}{.}\PY{n}{array}\PY{p}{(}\PY{p}{[}\PY{o}{\PYZhy{}}\PY{l+m+mi}{1}\PY{p}{,} \PY{l+m+mi}{2}\PY{p}{]}\PY{p}{)}\PY{p}{)}
         \PY{n+nb}{print}\PY{p}{(}\PY{l+s+s1}{\PYZsq{}}\PY{l+s+s1}{With theta = [\PYZhy{}1, 2]}\PY{l+s+se}{\PYZbs{}n}\PY{l+s+s1}{Cost computed = }\PY{l+s+si}{\PYZpc{}.2f}\PY{l+s+s1}{\PYZsq{}} \PY{o}{\PYZpc{}} \PY{n}{J}\PY{p}{)}
         \PY{n+nb}{print}\PY{p}{(}\PY{l+s+s1}{\PYZsq{}}\PY{l+s+s1}{Expected cost value (approximately) 54.24}\PY{l+s+s1}{\PYZsq{}}\PY{p}{)}
\end{Verbatim}

    \begin{Verbatim}[commandchars=\\\{\}]
With theta = [0, 0] 
Cost computed = 32.07
Expected cost value (approximately) 32.07

With theta = [-1, 2]
Cost computed = 54.24
Expected cost value (approximately) 54.24

    \end{Verbatim}

    \emph{You should now submit your solutions by executing the following
cell.}

    \begin{Verbatim}[commandchars=\\\{\}]
{\color{incolor}In [{\color{incolor}18}]:} \PY{n}{grader}\PY{p}{[}\PY{l+m+mi}{2}\PY{p}{]} \PY{o}{=} \PY{n}{computeCost}
         \PY{n}{grader}\PY{o}{.}\PY{n}{grade}\PY{p}{(}\PY{p}{)}
\end{Verbatim}

    \begin{Verbatim}[commandchars=\\\{\}]

Submitting Solutions | Programming Exercise linear-regression


    \end{Verbatim}

    \begin{Verbatim}[commandchars=\\\{\}]
Use token from last successful submission (miken024@live.com)? (Y/n):  y

    \end{Verbatim}

    \begin{Verbatim}[commandchars=\\\{\}]
                                  Part Name |     Score | Feedback
                                  --------- |     ----- | --------
                           Warm up exercise |  10 /  10 | Nice work!
          Computing Cost (for one variable) |  40 /  40 | Nice work!
        Gradient Descent (for one variable) |   0 /  50 | 
                      Feature Normalization |   0 /   0 | 
    Computing Cost (for multiple variables) |   0 /   0 | 
  Gradient Descent (for multiple variables) |   0 /   0 | 
                           Normal Equations |   0 /   0 | 
                                  --------------------------------
                                            |  50 / 100 |  


    \end{Verbatim}

     \#\#\#\# 2.2.4 Gradient descent

Next, you will complete a function which implements gradient descent.
The loop structure has been written for you, and you only need to supply
the updates to \(\theta\) within each iteration.

As you program, make sure you understand what you are trying to optimize
and what is being updated. Keep in mind that the cost \(J(\theta)\) is
parameterized by the vector \(\theta\), not \(X\) and \(y\). That is, we
minimize the value of \(J(\theta)\) by changing the values of the vector
\(\theta\), not by changing \(X\) or \(y\). Section \ref{section2} and
to the video lectures if you are uncertain. A good way to verify that
gradient descent is working correctly is to look at the value of
\(J(\theta)\) and check that it is decreasing with each step.

The starter code for the function \texttt{gradientDescent} calls
\texttt{computeCost} on every iteration and saves the cost to a
\texttt{python} list. Assuming you have implemented gradient descent and
\texttt{computeCost} correctly, your value of \(J(\theta)\) should never
increase, and should converge to a steady value by the end of the
algorithm.

\textbf{Vectors and matrices in \texttt{numpy}} - Important
implementation notes

A vector in \texttt{numpy} is a one dimensional array, for example
\texttt{np.array({[}1,\ 2,\ 3{]})} is a vector. A matrix in
\texttt{numpy} is a two dimensional array, for example
\texttt{np.array({[}{[}1,\ 2,\ 3{]},\ {[}4,\ 5,\ 6{]}{]})}. However, the
following is still considered a matrix
\texttt{np.array({[}{[}1,\ 2,\ 3{]}{]})} since it has two dimensions,
even if it has a shape of 1x3 (which looks like a vector).

Given the above, the function \texttt{np.dot} which we will use for all
matrix/vector multiplication has the following properties: - It always
performs inner products on vectors. If
\texttt{x=np.array({[}1,\ 2,\ 3{]})}, then \texttt{np.dot(x,\ x)} is a
scalar. - For matrix-vector multiplication, so if \(X\) is a
\(m\times n\) matrix and \(y\) is a vector of length \(m\), then the
operation \texttt{np.dot(y,\ X)} considers \(y\) as a \(1 \times m\)
vector. On the other hand, if \(y\) is a vector of length \(n\), then
the operation \texttt{np.dot(X,\ y)} considers \(y\) as a \(n \times 1\)
vector. - A vector can be promoted to a matrix using
\texttt{y{[}None{]}} or \texttt{{[}y{[}np.newaxis{]}}. That is, if
\texttt{y\ =\ np.array({[}1,\ 2,\ 3{]})} is a vector of size 3, then
\texttt{y{[}None,\ :{]}} is a matrix of shape \(1 \times 3\). We can use
\texttt{y{[}:,\ None{]}} to obtain a shape of \(3 \times 1\).

    \begin{Verbatim}[commandchars=\\\{\}]
{\color{incolor}In [{\color{incolor}19}]:} \PY{k}{def} \PY{n+nf}{gradientDescent}\PY{p}{(}\PY{n}{X}\PY{p}{,} \PY{n}{y}\PY{p}{,} \PY{n}{theta}\PY{p}{,} \PY{n}{alpha}\PY{p}{,} \PY{n}{num\PYZus{}iters}\PY{p}{)}\PY{p}{:}
             \PY{l+s+sd}{\PYZdq{}\PYZdq{}\PYZdq{}}
         \PY{l+s+sd}{    Performs gradient descent to learn `theta`. Updates theta by taking `num\PYZus{}iters`}
         \PY{l+s+sd}{    gradient steps with learning rate `alpha`.}
         \PY{l+s+sd}{    }
         \PY{l+s+sd}{    Parameters}
         \PY{l+s+sd}{    \PYZhy{}\PYZhy{}\PYZhy{}\PYZhy{}\PYZhy{}\PYZhy{}\PYZhy{}\PYZhy{}\PYZhy{}\PYZhy{}}
         \PY{l+s+sd}{    X : array\PYZus{}like}
         \PY{l+s+sd}{        The input dataset of shape (m x n+1).}
         \PY{l+s+sd}{    }
         \PY{l+s+sd}{    y : arra\PYZus{}like}
         \PY{l+s+sd}{        Value at given features. A vector of shape (m, ).}
         \PY{l+s+sd}{    }
         \PY{l+s+sd}{    theta : array\PYZus{}like}
         \PY{l+s+sd}{        Initial values for the linear regression parameters. }
         \PY{l+s+sd}{        A vector of shape (n+1, ).}
         \PY{l+s+sd}{    }
         \PY{l+s+sd}{    alpha : float}
         \PY{l+s+sd}{        The learning rate.}
         \PY{l+s+sd}{    }
         \PY{l+s+sd}{    num\PYZus{}iters : int}
         \PY{l+s+sd}{        The number of iterations for gradient descent. }
         \PY{l+s+sd}{    }
         \PY{l+s+sd}{    Returns}
         \PY{l+s+sd}{    \PYZhy{}\PYZhy{}\PYZhy{}\PYZhy{}\PYZhy{}\PYZhy{}\PYZhy{}}
         \PY{l+s+sd}{    theta : array\PYZus{}like}
         \PY{l+s+sd}{        The learned linear regression parameters. A vector of shape (n+1, ).}
         \PY{l+s+sd}{    }
         \PY{l+s+sd}{    J\PYZus{}history : list}
         \PY{l+s+sd}{        A python list for the values of the cost function after each iteration.}
         \PY{l+s+sd}{    }
         \PY{l+s+sd}{    Instructions}
         \PY{l+s+sd}{    \PYZhy{}\PYZhy{}\PYZhy{}\PYZhy{}\PYZhy{}\PYZhy{}\PYZhy{}\PYZhy{}\PYZhy{}\PYZhy{}\PYZhy{}\PYZhy{}}
         \PY{l+s+sd}{    Peform a single gradient step on the parameter vector theta.}
         
         \PY{l+s+sd}{    While debugging, it can be useful to print out the values of }
         \PY{l+s+sd}{    the cost function (computeCost) and gradient here.}
         \PY{l+s+sd}{    \PYZdq{}\PYZdq{}\PYZdq{}}
             \PY{c+c1}{\PYZsh{} Initialize some useful values}
             \PY{n}{m} \PY{o}{=} \PY{n}{y}\PY{o}{.}\PY{n}{shape}\PY{p}{[}\PY{l+m+mi}{0}\PY{p}{]}  \PY{c+c1}{\PYZsh{} number of training examples}
             
             \PY{c+c1}{\PYZsh{} make a copy of theta, to avoid changing the original array, since numpy arrays}
             \PY{c+c1}{\PYZsh{} are passed by reference to functions}
             \PY{n}{theta} \PY{o}{=} \PY{n}{theta}\PY{o}{.}\PY{n}{copy}\PY{p}{(}\PY{p}{)}
             
             \PY{n}{J\PYZus{}history} \PY{o}{=} \PY{p}{[}\PY{p}{]} \PY{c+c1}{\PYZsh{} Use a python list to save cost in every iteration}
             
             \PY{k}{for} \PY{n}{i} \PY{o+ow}{in} \PY{n+nb}{range}\PY{p}{(}\PY{n}{num\PYZus{}iters}\PY{p}{)}\PY{p}{:}
                 \PY{c+c1}{\PYZsh{} ==================== YOUR CODE HERE =================================}
                 \PY{n}{temp} \PY{o}{=} \PY{n}{np}\PY{o}{.}\PY{n}{dot}\PY{p}{(}\PY{n}{X}\PY{p}{,} \PY{n}{theta}\PY{p}{)} \PY{o}{\PYZhy{}} \PY{n}{y}
                 \PY{n}{temp} \PY{o}{=} \PY{n}{np}\PY{o}{.}\PY{n}{dot}\PY{p}{(}\PY{n}{X}\PY{o}{.}\PY{n}{T}\PY{p}{,} \PY{n}{temp}\PY{p}{)}
                 \PY{n}{theta} \PY{o}{=} \PY{n}{theta} \PY{o}{\PYZhy{}} \PY{p}{(}\PY{n}{alpha} \PY{o}{/} \PY{n}{m}\PY{p}{)} \PY{o}{*} \PY{n}{temp}
         
                 \PY{c+c1}{\PYZsh{} =====================================================================}
                 
                 \PY{c+c1}{\PYZsh{} save the cost J in every iteration}
                 \PY{n}{J\PYZus{}history}\PY{o}{.}\PY{n}{append}\PY{p}{(}\PY{n}{computeCost}\PY{p}{(}\PY{n}{X}\PY{p}{,} \PY{n}{y}\PY{p}{,} \PY{n}{theta}\PY{p}{)}\PY{p}{)}
             
             \PY{k}{return} \PY{n}{theta}\PY{p}{,} \PY{n}{J\PYZus{}history}
\end{Verbatim}

    After you are finished call the implemented \texttt{gradientDescent}
function and print the computed \(\theta\). We initialize the \(\theta\)
parameters to 0 and the learning rate \(\alpha\) to 0.01. Execute the
following cell to check your code.

    \begin{Verbatim}[commandchars=\\\{\}]
{\color{incolor}In [{\color{incolor}20}]:} \PY{c+c1}{\PYZsh{} initialize fitting parameters}
         \PY{n}{theta} \PY{o}{=} \PY{n}{np}\PY{o}{.}\PY{n}{zeros}\PY{p}{(}\PY{l+m+mi}{2}\PY{p}{)}
         
         \PY{c+c1}{\PYZsh{} some gradient descent settings}
         \PY{n}{iterations} \PY{o}{=} \PY{l+m+mi}{1500}
         \PY{n}{alpha} \PY{o}{=} \PY{l+m+mf}{0.01}
         
         \PY{n}{theta}\PY{p}{,} \PY{n}{J\PYZus{}history} \PY{o}{=} \PY{n}{gradientDescent}\PY{p}{(}\PY{n}{X} \PY{p}{,}\PY{n}{y}\PY{p}{,} \PY{n}{theta}\PY{p}{,} \PY{n}{alpha}\PY{p}{,} \PY{n}{iterations}\PY{p}{)}
         \PY{n+nb}{print}\PY{p}{(}\PY{l+s+s1}{\PYZsq{}}\PY{l+s+s1}{Theta found by gradient descent: }\PY{l+s+si}{\PYZob{}:.4f\PYZcb{}}\PY{l+s+s1}{, }\PY{l+s+si}{\PYZob{}:.4f\PYZcb{}}\PY{l+s+s1}{\PYZsq{}}\PY{o}{.}\PY{n}{format}\PY{p}{(}\PY{o}{*}\PY{n}{theta}\PY{p}{)}\PY{p}{)}
         \PY{n+nb}{print}\PY{p}{(}\PY{l+s+s1}{\PYZsq{}}\PY{l+s+s1}{Expected theta values (approximately): [\PYZhy{}3.6303, 1.1664]}\PY{l+s+s1}{\PYZsq{}}\PY{p}{)}
\end{Verbatim}

    \begin{Verbatim}[commandchars=\\\{\}]
Theta found by gradient descent: -3.6303, 1.1664
Expected theta values (approximately): [-3.6303, 1.1664]

    \end{Verbatim}

    We will use your final parameters to plot the linear fit. The results
should look like the following figure.

\begin{figure}
\centering
\includegraphics{Figures/regression_result.png}
\caption{}
\end{figure}

    \begin{Verbatim}[commandchars=\\\{\}]
{\color{incolor}In [{\color{incolor}21}]:} \PY{c+c1}{\PYZsh{} plot the linear fit}
         \PY{n}{plotData}\PY{p}{(}\PY{n}{X}\PY{p}{[}\PY{p}{:}\PY{p}{,} \PY{l+m+mi}{1}\PY{p}{]}\PY{p}{,} \PY{n}{y}\PY{p}{)}
         \PY{n}{pyplot}\PY{o}{.}\PY{n}{plot}\PY{p}{(}\PY{n}{X}\PY{p}{[}\PY{p}{:}\PY{p}{,} \PY{l+m+mi}{1}\PY{p}{]}\PY{p}{,} \PY{n}{np}\PY{o}{.}\PY{n}{dot}\PY{p}{(}\PY{n}{X}\PY{p}{,} \PY{n}{theta}\PY{p}{)}\PY{p}{,} \PY{l+s+s1}{\PYZsq{}}\PY{l+s+s1}{\PYZhy{}}\PY{l+s+s1}{\PYZsq{}}\PY{p}{)}
         \PY{n}{pyplot}\PY{o}{.}\PY{n}{legend}\PY{p}{(}\PY{p}{[}\PY{l+s+s1}{\PYZsq{}}\PY{l+s+s1}{Training data}\PY{l+s+s1}{\PYZsq{}}\PY{p}{,} \PY{l+s+s1}{\PYZsq{}}\PY{l+s+s1}{Linear regression}\PY{l+s+s1}{\PYZsq{}}\PY{p}{]}\PY{p}{)}\PY{p}{;}
\end{Verbatim}

    \begin{center}
    \adjustimage{max size={0.9\linewidth}{0.9\paperheight}}{output_30_0.png}
    \end{center}
    { \hspace*{\fill} \\}
    
    Your final values for \(\theta\) will also be used to make predictions
on profits in areas of 35,000 and 70,000 people.

Note the way that the following lines use matrix multiplication, rather
than explicit summation or looping, to calculate the predictions. This
is an example of code vectorization in \texttt{numpy}.

Note that the first argument to the \texttt{numpy} function \texttt{dot}
is a python list. \texttt{numpy} can internally converts \textbf{valid}
python lists to numpy arrays when explicitly provided as arguments to
\texttt{numpy} functions.

    \begin{Verbatim}[commandchars=\\\{\}]
{\color{incolor}In [{\color{incolor}22}]:} \PY{c+c1}{\PYZsh{} Predict values for population sizes of 35,000 and 70,000}
         \PY{n}{predict1} \PY{o}{=} \PY{n}{np}\PY{o}{.}\PY{n}{dot}\PY{p}{(}\PY{p}{[}\PY{l+m+mi}{1}\PY{p}{,} \PY{l+m+mf}{3.5}\PY{p}{]}\PY{p}{,} \PY{n}{theta}\PY{p}{)}
         \PY{n+nb}{print}\PY{p}{(}\PY{l+s+s1}{\PYZsq{}}\PY{l+s+s1}{For population = 35,000, we predict a profit of }\PY{l+s+si}{\PYZob{}:.2f\PYZcb{}}\PY{l+s+se}{\PYZbs{}n}\PY{l+s+s1}{\PYZsq{}}\PY{o}{.}\PY{n}{format}\PY{p}{(}\PY{n}{predict1}\PY{o}{*}\PY{l+m+mi}{10000}\PY{p}{)}\PY{p}{)}
         
         \PY{n}{predict2} \PY{o}{=} \PY{n}{np}\PY{o}{.}\PY{n}{dot}\PY{p}{(}\PY{p}{[}\PY{l+m+mi}{1}\PY{p}{,} \PY{l+m+mi}{7}\PY{p}{]}\PY{p}{,} \PY{n}{theta}\PY{p}{)}
         \PY{n+nb}{print}\PY{p}{(}\PY{l+s+s1}{\PYZsq{}}\PY{l+s+s1}{For population = 70,000, we predict a profit of }\PY{l+s+si}{\PYZob{}:.2f\PYZcb{}}\PY{l+s+se}{\PYZbs{}n}\PY{l+s+s1}{\PYZsq{}}\PY{o}{.}\PY{n}{format}\PY{p}{(}\PY{n}{predict2}\PY{o}{*}\PY{l+m+mi}{10000}\PY{p}{)}\PY{p}{)}
\end{Verbatim}

    \begin{Verbatim}[commandchars=\\\{\}]
For population = 35,000, we predict a profit of 4519.77

For population = 70,000, we predict a profit of 45342.45


    \end{Verbatim}

    \emph{You should now submit your solutions by executing the next cell.}

    \begin{Verbatim}[commandchars=\\\{\}]
{\color{incolor}In [{\color{incolor}23}]:} \PY{n}{grader}\PY{p}{[}\PY{l+m+mi}{3}\PY{p}{]} \PY{o}{=} \PY{n}{gradientDescent}
         \PY{n}{grader}\PY{o}{.}\PY{n}{grade}\PY{p}{(}\PY{p}{)}
\end{Verbatim}

    \begin{Verbatim}[commandchars=\\\{\}]

Submitting Solutions | Programming Exercise linear-regression


    \end{Verbatim}

    \begin{Verbatim}[commandchars=\\\{\}]
Use token from last successful submission (miken024@live.com)? (Y/n):  y

    \end{Verbatim}

    \begin{Verbatim}[commandchars=\\\{\}]
                                  Part Name |     Score | Feedback
                                  --------- |     ----- | --------
                           Warm up exercise |  10 /  10 | Nice work!
          Computing Cost (for one variable) |  40 /  40 | Nice work!
        Gradient Descent (for one variable) |  50 /  50 | Nice work!
                      Feature Normalization |   0 /   0 | 
    Computing Cost (for multiple variables) |   0 /   0 | 
  Gradient Descent (for multiple variables) |   0 /   0 | 
                           Normal Equations |   0 /   0 | 
                                  --------------------------------
                                            | 100 / 100 |  


    \end{Verbatim}

    \subsubsection{\texorpdfstring{2.4 Visualizing
\(J(\theta)\)}{2.4 Visualizing J(\textbackslash{}theta)}}\label{visualizing-jtheta}

To understand the cost function \(J(\theta)\) better, you will now plot
the cost over a 2-dimensional grid of \(\theta_0\) and \(\theta_1\)
values. You will not need to code anything new for this part, but you
should understand how the code you have written already is creating
these images.

In the next cell, the code is set up to calculate \(J(\theta)\) over a
grid of values using the \texttt{computeCost} function that you wrote.
After executing the following cell, you will have a 2-D array of
\(J(\theta)\) values. Then, those values are used to produce surface and
contour plots of \(J(\theta)\) using the matplotlib
\texttt{plot\_surface} and \texttt{contourf} functions. The plots should
look something like the following:

\begin{figure}
\centering
\includegraphics{Figures/cost_function.png}
\caption{}
\end{figure}

The purpose of these graphs is to show you how \(J(\theta)\) varies with
changes in \(\theta_0\) and \(\theta_1\). The cost function
\(J(\theta)\) is bowl-shaped and has a global minimum. (This is easier
to see in the contour plot than in the 3D surface plot). This minimum is
the optimal point for \(\theta_0\) and \(\theta_1\), and each step of
gradient descent moves closer to this point.

    \begin{Verbatim}[commandchars=\\\{\}]
{\color{incolor}In [{\color{incolor}24}]:} \PY{c+c1}{\PYZsh{} grid over which we will calculate J}
         \PY{n}{theta0\PYZus{}vals} \PY{o}{=} \PY{n}{np}\PY{o}{.}\PY{n}{linspace}\PY{p}{(}\PY{o}{\PYZhy{}}\PY{l+m+mi}{10}\PY{p}{,} \PY{l+m+mi}{10}\PY{p}{,} \PY{l+m+mi}{100}\PY{p}{)}
         \PY{n}{theta1\PYZus{}vals} \PY{o}{=} \PY{n}{np}\PY{o}{.}\PY{n}{linspace}\PY{p}{(}\PY{o}{\PYZhy{}}\PY{l+m+mi}{1}\PY{p}{,} \PY{l+m+mi}{4}\PY{p}{,} \PY{l+m+mi}{100}\PY{p}{)}
         
         \PY{c+c1}{\PYZsh{} initialize J\PYZus{}vals to a matrix of 0\PYZsq{}s}
         \PY{n}{J\PYZus{}vals} \PY{o}{=} \PY{n}{np}\PY{o}{.}\PY{n}{zeros}\PY{p}{(}\PY{p}{(}\PY{n}{theta0\PYZus{}vals}\PY{o}{.}\PY{n}{shape}\PY{p}{[}\PY{l+m+mi}{0}\PY{p}{]}\PY{p}{,} \PY{n}{theta1\PYZus{}vals}\PY{o}{.}\PY{n}{shape}\PY{p}{[}\PY{l+m+mi}{0}\PY{p}{]}\PY{p}{)}\PY{p}{)}
         
         \PY{c+c1}{\PYZsh{} Fill out J\PYZus{}vals}
         \PY{k}{for} \PY{n}{i}\PY{p}{,} \PY{n}{theta0} \PY{o+ow}{in} \PY{n+nb}{enumerate}\PY{p}{(}\PY{n}{theta0\PYZus{}vals}\PY{p}{)}\PY{p}{:}
             \PY{k}{for} \PY{n}{j}\PY{p}{,} \PY{n}{theta1} \PY{o+ow}{in} \PY{n+nb}{enumerate}\PY{p}{(}\PY{n}{theta1\PYZus{}vals}\PY{p}{)}\PY{p}{:}
                 \PY{n}{J\PYZus{}vals}\PY{p}{[}\PY{n}{i}\PY{p}{,} \PY{n}{j}\PY{p}{]} \PY{o}{=} \PY{n}{computeCost}\PY{p}{(}\PY{n}{X}\PY{p}{,} \PY{n}{y}\PY{p}{,} \PY{p}{[}\PY{n}{theta0}\PY{p}{,} \PY{n}{theta1}\PY{p}{]}\PY{p}{)}
                 
         \PY{c+c1}{\PYZsh{} Because of the way meshgrids work in the surf command, we need to}
         \PY{c+c1}{\PYZsh{} transpose J\PYZus{}vals before calling surf, or else the axes will be flipped}
         \PY{n}{J\PYZus{}vals} \PY{o}{=} \PY{n}{J\PYZus{}vals}\PY{o}{.}\PY{n}{T}
         
         \PY{c+c1}{\PYZsh{} surface plot}
         \PY{n}{fig} \PY{o}{=} \PY{n}{pyplot}\PY{o}{.}\PY{n}{figure}\PY{p}{(}\PY{n}{figsize}\PY{o}{=}\PY{p}{(}\PY{l+m+mi}{12}\PY{p}{,} \PY{l+m+mi}{5}\PY{p}{)}\PY{p}{)}
         \PY{n}{ax} \PY{o}{=} \PY{n}{fig}\PY{o}{.}\PY{n}{add\PYZus{}subplot}\PY{p}{(}\PY{l+m+mi}{121}\PY{p}{,} \PY{n}{projection}\PY{o}{=}\PY{l+s+s1}{\PYZsq{}}\PY{l+s+s1}{3d}\PY{l+s+s1}{\PYZsq{}}\PY{p}{)}
         \PY{n}{ax}\PY{o}{.}\PY{n}{plot\PYZus{}surface}\PY{p}{(}\PY{n}{theta0\PYZus{}vals}\PY{p}{,} \PY{n}{theta1\PYZus{}vals}\PY{p}{,} \PY{n}{J\PYZus{}vals}\PY{p}{,} \PY{n}{cmap}\PY{o}{=}\PY{l+s+s1}{\PYZsq{}}\PY{l+s+s1}{viridis}\PY{l+s+s1}{\PYZsq{}}\PY{p}{)}
         \PY{n}{pyplot}\PY{o}{.}\PY{n}{xlabel}\PY{p}{(}\PY{l+s+s1}{\PYZsq{}}\PY{l+s+s1}{theta0}\PY{l+s+s1}{\PYZsq{}}\PY{p}{)}
         \PY{n}{pyplot}\PY{o}{.}\PY{n}{ylabel}\PY{p}{(}\PY{l+s+s1}{\PYZsq{}}\PY{l+s+s1}{theta1}\PY{l+s+s1}{\PYZsq{}}\PY{p}{)}
         \PY{n}{pyplot}\PY{o}{.}\PY{n}{title}\PY{p}{(}\PY{l+s+s1}{\PYZsq{}}\PY{l+s+s1}{Surface}\PY{l+s+s1}{\PYZsq{}}\PY{p}{)}
         
         \PY{c+c1}{\PYZsh{} contour plot}
         \PY{c+c1}{\PYZsh{} Plot J\PYZus{}vals as 15 contours spaced logarithmically between 0.01 and 100}
         \PY{n}{ax} \PY{o}{=} \PY{n}{pyplot}\PY{o}{.}\PY{n}{subplot}\PY{p}{(}\PY{l+m+mi}{122}\PY{p}{)}
         \PY{n}{pyplot}\PY{o}{.}\PY{n}{contour}\PY{p}{(}\PY{n}{theta0\PYZus{}vals}\PY{p}{,} \PY{n}{theta1\PYZus{}vals}\PY{p}{,} \PY{n}{J\PYZus{}vals}\PY{p}{,} \PY{n}{linewidths}\PY{o}{=}\PY{l+m+mi}{2}\PY{p}{,} \PY{n}{cmap}\PY{o}{=}\PY{l+s+s1}{\PYZsq{}}\PY{l+s+s1}{viridis}\PY{l+s+s1}{\PYZsq{}}\PY{p}{,} \PY{n}{levels}\PY{o}{=}\PY{n}{np}\PY{o}{.}\PY{n}{logspace}\PY{p}{(}\PY{o}{\PYZhy{}}\PY{l+m+mi}{2}\PY{p}{,} \PY{l+m+mi}{3}\PY{p}{,} \PY{l+m+mi}{20}\PY{p}{)}\PY{p}{)}
         \PY{n}{pyplot}\PY{o}{.}\PY{n}{xlabel}\PY{p}{(}\PY{l+s+s1}{\PYZsq{}}\PY{l+s+s1}{theta0}\PY{l+s+s1}{\PYZsq{}}\PY{p}{)}
         \PY{n}{pyplot}\PY{o}{.}\PY{n}{ylabel}\PY{p}{(}\PY{l+s+s1}{\PYZsq{}}\PY{l+s+s1}{theta1}\PY{l+s+s1}{\PYZsq{}}\PY{p}{)}
         \PY{n}{pyplot}\PY{o}{.}\PY{n}{plot}\PY{p}{(}\PY{n}{theta}\PY{p}{[}\PY{l+m+mi}{0}\PY{p}{]}\PY{p}{,} \PY{n}{theta}\PY{p}{[}\PY{l+m+mi}{1}\PY{p}{]}\PY{p}{,} \PY{l+s+s1}{\PYZsq{}}\PY{l+s+s1}{ro}\PY{l+s+s1}{\PYZsq{}}\PY{p}{,} \PY{n}{ms}\PY{o}{=}\PY{l+m+mi}{10}\PY{p}{,} \PY{n}{lw}\PY{o}{=}\PY{l+m+mi}{2}\PY{p}{)}
         \PY{n}{pyplot}\PY{o}{.}\PY{n}{title}\PY{p}{(}\PY{l+s+s1}{\PYZsq{}}\PY{l+s+s1}{Contour, showing minimum}\PY{l+s+s1}{\PYZsq{}}\PY{p}{)}
         \PY{k}{pass}
\end{Verbatim}

    \begin{center}
    \adjustimage{max size={0.9\linewidth}{0.9\paperheight}}{output_36_0.png}
    \end{center}
    { \hspace*{\fill} \\}
    
    \subsection{Optional Exercises}\label{optional-exercises}

If you have successfully completed the material above, congratulations!
You now understand linear regression and should able to start using it
on your own datasets.

For the rest of this programming exercise, we have included the
following optional exercises. These exercises will help you gain a
deeper understanding of the material, and if you are able to do so, we
encourage you to complete them as well. You can still submit your
solutions to these exercises to check if your answers are correct.

\subsection{3 Linear regression with multiple
variables}\label{linear-regression-with-multiple-variables}

In this part, you will implement linear regression with multiple
variables to predict the prices of houses. Suppose you are selling your
house and you want to know what a good market price would be. One way to
do this is to first collect information on recent houses sold and make a
model of housing prices.

The file \texttt{Data/ex1data2.txt} contains a training set of housing
prices in Portland, Oregon. The first column is the size of the house
(in square feet), the second column is the number of bedrooms, and the
third column is the price of the house.

 \#\#\# 3.1 Feature Normalization

We start by loading and displaying some values from this dataset. By
looking at the values, note that house sizes are about 1000 times the
number of bedrooms. When features differ by orders of magnitude, first
performing feature scaling can make gradient descent converge much more
quickly.

    \begin{Verbatim}[commandchars=\\\{\}]
{\color{incolor}In [{\color{incolor}25}]:} \PY{c+c1}{\PYZsh{} Load data}
         \PY{n}{data} \PY{o}{=} \PY{n}{np}\PY{o}{.}\PY{n}{loadtxt}\PY{p}{(}\PY{n}{os}\PY{o}{.}\PY{n}{path}\PY{o}{.}\PY{n}{join}\PY{p}{(}\PY{l+s+s1}{\PYZsq{}}\PY{l+s+s1}{Data}\PY{l+s+s1}{\PYZsq{}}\PY{p}{,} \PY{l+s+s1}{\PYZsq{}}\PY{l+s+s1}{ex1data2.txt}\PY{l+s+s1}{\PYZsq{}}\PY{p}{)}\PY{p}{,} \PY{n}{delimiter}\PY{o}{=}\PY{l+s+s1}{\PYZsq{}}\PY{l+s+s1}{,}\PY{l+s+s1}{\PYZsq{}}\PY{p}{)}
         \PY{n}{X} \PY{o}{=} \PY{n}{data}\PY{p}{[}\PY{p}{:}\PY{p}{,} \PY{p}{:}\PY{l+m+mi}{2}\PY{p}{]}
         \PY{n}{y} \PY{o}{=} \PY{n}{data}\PY{p}{[}\PY{p}{:}\PY{p}{,} \PY{l+m+mi}{2}\PY{p}{]}
         \PY{n}{m} \PY{o}{=} \PY{n}{y}\PY{o}{.}\PY{n}{size}
         
         \PY{c+c1}{\PYZsh{} print out some data points}
         \PY{n+nb}{print}\PY{p}{(}\PY{l+s+s1}{\PYZsq{}}\PY{l+s+si}{\PYZob{}:\PYZgt{}8s\PYZcb{}}\PY{l+s+si}{\PYZob{}:\PYZgt{}8s\PYZcb{}}\PY{l+s+si}{\PYZob{}:\PYZgt{}10s\PYZcb{}}\PY{l+s+s1}{\PYZsq{}}\PY{o}{.}\PY{n}{format}\PY{p}{(}\PY{l+s+s1}{\PYZsq{}}\PY{l+s+s1}{X[:,0]}\PY{l+s+s1}{\PYZsq{}}\PY{p}{,} \PY{l+s+s1}{\PYZsq{}}\PY{l+s+s1}{X[:, 1]}\PY{l+s+s1}{\PYZsq{}}\PY{p}{,} \PY{l+s+s1}{\PYZsq{}}\PY{l+s+s1}{y}\PY{l+s+s1}{\PYZsq{}}\PY{p}{)}\PY{p}{)}
         \PY{n+nb}{print}\PY{p}{(}\PY{l+s+s1}{\PYZsq{}}\PY{l+s+s1}{\PYZhy{}}\PY{l+s+s1}{\PYZsq{}}\PY{o}{*}\PY{l+m+mi}{26}\PY{p}{)}
         \PY{k}{for} \PY{n}{i} \PY{o+ow}{in} \PY{n+nb}{range}\PY{p}{(}\PY{l+m+mi}{10}\PY{p}{)}\PY{p}{:}
             \PY{n+nb}{print}\PY{p}{(}\PY{l+s+s1}{\PYZsq{}}\PY{l+s+si}{\PYZob{}:8.0f\PYZcb{}}\PY{l+s+si}{\PYZob{}:8.0f\PYZcb{}}\PY{l+s+si}{\PYZob{}:10.0f\PYZcb{}}\PY{l+s+s1}{\PYZsq{}}\PY{o}{.}\PY{n}{format}\PY{p}{(}\PY{n}{X}\PY{p}{[}\PY{n}{i}\PY{p}{,} \PY{l+m+mi}{0}\PY{p}{]}\PY{p}{,} \PY{n}{X}\PY{p}{[}\PY{n}{i}\PY{p}{,} \PY{l+m+mi}{1}\PY{p}{]}\PY{p}{,} \PY{n}{y}\PY{p}{[}\PY{n}{i}\PY{p}{]}\PY{p}{)}\PY{p}{)}
\end{Verbatim}

    \begin{Verbatim}[commandchars=\\\{\}]
  X[:,0] X[:, 1]         y
--------------------------
    2104       3    399900
    1600       3    329900
    2400       3    369000
    1416       2    232000
    3000       4    539900
    1985       4    299900
    1534       3    314900
    1427       3    198999
    1380       3    212000
    1494       3    242500

    \end{Verbatim}

    Your task here is to complete the code in \texttt{featureNormalize}
function: - Subtract the mean value of each feature from the dataset. -
After subtracting the mean, additionally scale (divide) the feature
values by their respective ``standard deviations.''

The standard deviation is a way of measuring how much variation there is
in the range of values of a particular feature (most data points will
lie within ±2 standard deviations of the mean); this is an alternative
to taking the range of values (max-min). In \texttt{numpy}, you can use
the \texttt{std} function to compute the standard deviation.

For example, the quantity \texttt{X{[}:,\ 0{]}} contains all the values
of \(x_1\) (house sizes) in the training set, so
\texttt{np.std(X{[}:,\ 0{]})} computes the standard deviation of the
house sizes. At the time that the function \texttt{featureNormalize} is
called, the extra column of 1's corresponding to \(x_0 = 1\) has not yet
been added to \(X\).

You will do this for all the features and your code should work with
datasets of all sizes (any number of features / examples). Note that
each column of the matrix \(X\) corresponds to one feature.

\textbf{Implementation Note:} When normalizing the features, it is
important to store the values used for normalization - the mean value
and the standard deviation used for the computations. After learning the
parameters from the model, we often want to predict the prices of houses
we have not seen before. Given a new x value (living room area and
number of bedrooms), we must first normalize x using the mean and
standard deviation that we had previously computed from the training
set.

    \begin{Verbatim}[commandchars=\\\{\}]
{\color{incolor}In [{\color{incolor}26}]:} \PY{k}{def}  \PY{n+nf}{featureNormalize}\PY{p}{(}\PY{n}{X}\PY{p}{)}\PY{p}{:}
             \PY{l+s+sd}{\PYZdq{}\PYZdq{}\PYZdq{}}
         \PY{l+s+sd}{    Normalizes the features in X. returns a normalized version of X where}
         \PY{l+s+sd}{    the mean value of each feature is 0 and the standard deviation}
         \PY{l+s+sd}{    is 1. This is often a good preprocessing step to do when working with}
         \PY{l+s+sd}{    learning algorithms.}
         \PY{l+s+sd}{    }
         \PY{l+s+sd}{    Parameters}
         \PY{l+s+sd}{    \PYZhy{}\PYZhy{}\PYZhy{}\PYZhy{}\PYZhy{}\PYZhy{}\PYZhy{}\PYZhy{}\PYZhy{}\PYZhy{}}
         \PY{l+s+sd}{    X : array\PYZus{}like}
         \PY{l+s+sd}{        The dataset of shape (m x n).}
         \PY{l+s+sd}{    }
         \PY{l+s+sd}{    Returns}
         \PY{l+s+sd}{    \PYZhy{}\PYZhy{}\PYZhy{}\PYZhy{}\PYZhy{}\PYZhy{}\PYZhy{}}
         \PY{l+s+sd}{    X\PYZus{}norm : array\PYZus{}like}
         \PY{l+s+sd}{        The normalized dataset of shape (m x n).}
         \PY{l+s+sd}{    }
         \PY{l+s+sd}{    Instructions}
         \PY{l+s+sd}{    \PYZhy{}\PYZhy{}\PYZhy{}\PYZhy{}\PYZhy{}\PYZhy{}\PYZhy{}\PYZhy{}\PYZhy{}\PYZhy{}\PYZhy{}\PYZhy{}}
         \PY{l+s+sd}{    First, for each feature dimension, compute the mean of the feature}
         \PY{l+s+sd}{    and subtract it from the dataset, storing the mean value in mu. }
         \PY{l+s+sd}{    Next, compute the  standard deviation of each feature and divide}
         \PY{l+s+sd}{    each feature by it\PYZsq{}s standard deviation, storing the standard deviation }
         \PY{l+s+sd}{    in sigma. }
         \PY{l+s+sd}{    }
         \PY{l+s+sd}{    Note that X is a matrix where each column is a feature and each row is}
         \PY{l+s+sd}{    an example. You needto perform the normalization separately for each feature. }
         \PY{l+s+sd}{    }
         \PY{l+s+sd}{    Hint}
         \PY{l+s+sd}{    \PYZhy{}\PYZhy{}\PYZhy{}\PYZhy{}}
         \PY{l+s+sd}{    You might find the \PYZsq{}np.mean\PYZsq{} and \PYZsq{}np.std\PYZsq{} functions useful.}
         \PY{l+s+sd}{    \PYZdq{}\PYZdq{}\PYZdq{}}
             \PY{c+c1}{\PYZsh{} You need to set these values correctly}
             \PY{n}{X\PYZus{}norm} \PY{o}{=} \PY{n}{X}\PY{o}{.}\PY{n}{copy}\PY{p}{(}\PY{p}{)}
             \PY{n}{mu}    \PY{o}{=} \PY{n}{np}\PY{o}{.}\PY{n}{zeros}\PY{p}{(}\PY{p}{(}\PY{l+m+mi}{1}\PY{p}{,} \PY{n}{X}\PY{o}{.}\PY{n}{shape}\PY{p}{[}\PY{l+m+mi}{1}\PY{p}{]}\PY{p}{)}\PY{p}{)}
             \PY{n}{sigma} \PY{o}{=} \PY{n}{np}\PY{o}{.}\PY{n}{zeros}\PY{p}{(}\PY{p}{(}\PY{l+m+mi}{1}\PY{p}{,} \PY{n}{X}\PY{o}{.}\PY{n}{shape}\PY{p}{[}\PY{l+m+mi}{1}\PY{p}{]}\PY{p}{)}\PY{p}{)}
         
             \PY{c+c1}{\PYZsh{} =========================== YOUR CODE HERE =====================}
             \PY{k}{for} \PY{n}{i} \PY{o+ow}{in} \PY{n+nb}{range} \PY{p}{(}\PY{n}{X}\PY{o}{.}\PY{n}{shape}\PY{p}{[}\PY{l+m+mi}{1}\PY{p}{]}\PY{p}{)}\PY{p}{:}
                 \PY{n}{mu}\PY{p}{[}\PY{p}{:}\PY{p}{,}\PY{n}{i}\PY{p}{]} \PY{o}{=} \PY{n}{np}\PY{o}{.}\PY{n}{mean}\PY{p}{(}\PY{n}{X}\PY{p}{[}\PY{p}{:}\PY{p}{,}\PY{n}{i}\PY{p}{]}\PY{p}{)}
                 \PY{n}{sigma}\PY{p}{[}\PY{p}{:}\PY{p}{,}\PY{n}{i}\PY{p}{]} \PY{o}{=} \PY{n}{np}\PY{o}{.}\PY{n}{std}\PY{p}{(}\PY{n}{X}\PY{p}{[}\PY{p}{:}\PY{p}{,}\PY{n}{i}\PY{p}{]}\PY{p}{)}
                 \PY{n}{X\PYZus{}norm}\PY{p}{[}\PY{p}{:}\PY{p}{,}\PY{n}{i}\PY{p}{]} \PY{o}{=} \PY{p}{(}\PY{n}{X}\PY{p}{[}\PY{p}{:}\PY{p}{,}\PY{n}{i}\PY{p}{]} \PY{o}{\PYZhy{}} \PY{n}{mu}\PY{p}{[}\PY{p}{:}\PY{p}{,}\PY{n}{i}\PY{p}{]}\PY{p}{)} \PY{o}{/} \PY{n}{sigma}\PY{p}{[}\PY{p}{:}\PY{p}{,}\PY{n}{i}\PY{p}{]}
             
             \PY{c+c1}{\PYZsh{} ================================================================}
             \PY{k}{return} \PY{n}{X\PYZus{}norm}\PY{p}{,} \PY{n}{mu}\PY{p}{,} \PY{n}{sigma}
\end{Verbatim}

    \begin{Verbatim}[commandchars=\\\{\}]
{\color{incolor}In [{\color{incolor}27}]:} \PY{n}{Execute} \PY{n}{the} \PY{n+nb}{next} \PY{n}{cell} \PY{n}{to} \PY{n}{run} \PY{n}{the} \PY{n}{implemented} \PY{err}{`}\PY{n}{featureNormalize}\PY{err}{`} \PY{n}{function}\PY{o}{.}
\end{Verbatim}

    \begin{Verbatim}[commandchars=\\\{\}]

          File "<ipython-input-27-6ad6b2611070>", line 1
        Execute the next cell to run the implemented `featureNormalize` function.
                  \^{}
    SyntaxError: invalid syntax
    

    \end{Verbatim}

    \begin{Verbatim}[commandchars=\\\{\}]
{\color{incolor}In [{\color{incolor}28}]:} \PY{c+c1}{\PYZsh{} call featureNormalize on the loaded data}
         \PY{n}{X\PYZus{}norm}\PY{p}{,} \PY{n}{mu}\PY{p}{,} \PY{n}{sigma} \PY{o}{=} \PY{n}{featureNormalize}\PY{p}{(}\PY{n}{X}\PY{p}{)}
         
         \PY{n+nb}{print}\PY{p}{(}\PY{l+s+s1}{\PYZsq{}}\PY{l+s+s1}{Computed mean:}\PY{l+s+s1}{\PYZsq{}}\PY{p}{,} \PY{n}{mu}\PY{p}{)}
         \PY{n+nb}{print}\PY{p}{(}\PY{l+s+s1}{\PYZsq{}}\PY{l+s+s1}{Computed standard deviation:}\PY{l+s+s1}{\PYZsq{}}\PY{p}{,} \PY{n}{sigma}\PY{p}{)}
\end{Verbatim}

    \begin{Verbatim}[commandchars=\\\{\}]
Computed mean: [[2000.68085106    3.17021277]]
Computed standard deviation: [[7.86202619e+02 7.52842809e-01]]

    \end{Verbatim}

    \begin{Verbatim}[commandchars=\\\{\}]
{\color{incolor}In [{\color{incolor}29}]:} \PY{n}{grader}\PY{p}{[}\PY{l+m+mi}{4}\PY{p}{]} \PY{o}{=} \PY{n}{featureNormalize}
         \PY{n}{grader}\PY{o}{.}\PY{n}{grade}\PY{p}{(}\PY{p}{)}
\end{Verbatim}

    \begin{Verbatim}[commandchars=\\\{\}]

Submitting Solutions | Programming Exercise linear-regression


    \end{Verbatim}

    \begin{Verbatim}[commandchars=\\\{\}]
Use token from last successful submission (miken024@live.com)? (Y/n):  y

    \end{Verbatim}

    \begin{Verbatim}[commandchars=\\\{\}]
                                  Part Name |     Score | Feedback
                                  --------- |     ----- | --------
                           Warm up exercise |  10 /  10 | Nice work!
          Computing Cost (for one variable) |  40 /  40 | Nice work!
        Gradient Descent (for one variable) |  50 /  50 | Nice work!
                      Feature Normalization |   0 /   0 | Nice work!
    Computing Cost (for multiple variables) |   0 /   0 | 
  Gradient Descent (for multiple variables) |   0 /   0 | 
                           Normal Equations |   0 /   0 | 
                                  --------------------------------
                                            | 100 / 100 |  


    \end{Verbatim}

    After the \texttt{featureNormalize} function is tested, we now add the
intercept term to \texttt{X\_norm}:

    \begin{Verbatim}[commandchars=\\\{\}]
{\color{incolor}In [{\color{incolor}30}]:} \PY{c+c1}{\PYZsh{} Add intercept term to X}
         \PY{n}{X} \PY{o}{=} \PY{n}{np}\PY{o}{.}\PY{n}{concatenate}\PY{p}{(}\PY{p}{[}\PY{n}{np}\PY{o}{.}\PY{n}{ones}\PY{p}{(}\PY{p}{(}\PY{n}{m}\PY{p}{,} \PY{l+m+mi}{1}\PY{p}{)}\PY{p}{)}\PY{p}{,} \PY{n}{X\PYZus{}norm}\PY{p}{]}\PY{p}{,} \PY{n}{axis}\PY{o}{=}\PY{l+m+mi}{1}\PY{p}{)}
\end{Verbatim}

     \#\#\# 3.2 Gradient Descent

Previously, you implemented gradient descent on a univariate regression
problem. The only difference now is that there is one more feature in
the matrix \(X\). The hypothesis function and the batch gradient descent
update rule remain unchanged.

You should complete the code for the functions \texttt{computeCostMulti}
and \texttt{gradientDescentMulti} to implement the cost function and
gradient descent for linear regression with multiple variables. If your
code in the previous part (single variable) already supports multiple
variables, you can use it here too. Make sure your code supports any
number of features and is well-vectorized. You can use the
\texttt{shape} property of \texttt{numpy} arrays to find out how many
features are present in the dataset.

\textbf{Implementation Note:} In the multivariate case, the cost
function can also be written in the following vectorized form:

\[ J(\theta) = \frac{1}{2m}(X\theta - \vec{y})^T(X\theta - \vec{y}) \]

where

\[ X = \begin{pmatrix}
          - (x^{(1)})^T - \\
          - (x^{(2)})^T - \\
          \vdots \\
          - (x^{(m)})^T - \\ \\
        \end{pmatrix} \qquad \mathbf{y} = \begin{bmatrix} y^{(1)} \\ y^{(2)} \\ \vdots \\ y^{(m)} \\\end{bmatrix}\]

the vectorized version is efficient when you are working with numerical
computing tools like \texttt{numpy}. If you are an expert with matrix
operations, you can prove to yourself that the two forms are equivalent.

    \begin{Verbatim}[commandchars=\\\{\}]
{\color{incolor}In [{\color{incolor}31}]:} \PY{k}{def} \PY{n+nf}{computeCostMulti}\PY{p}{(}\PY{n}{X}\PY{p}{,} \PY{n}{y}\PY{p}{,} \PY{n}{theta}\PY{p}{)}\PY{p}{:}
             \PY{l+s+sd}{\PYZdq{}\PYZdq{}\PYZdq{}}
         \PY{l+s+sd}{    Compute cost for linear regression with multiple variables.}
         \PY{l+s+sd}{    Computes the cost of using theta as the parameter for linear regression to fit the data points in X and y.}
         \PY{l+s+sd}{    }
         \PY{l+s+sd}{    Parameters}
         \PY{l+s+sd}{    \PYZhy{}\PYZhy{}\PYZhy{}\PYZhy{}\PYZhy{}\PYZhy{}\PYZhy{}\PYZhy{}\PYZhy{}\PYZhy{}}
         \PY{l+s+sd}{    X : array\PYZus{}like}
         \PY{l+s+sd}{        The dataset of shape (m x n+1).}
         \PY{l+s+sd}{    }
         \PY{l+s+sd}{    y : array\PYZus{}like}
         \PY{l+s+sd}{        A vector of shape (m, ) for the values at a given data point.}
         \PY{l+s+sd}{    }
         \PY{l+s+sd}{    theta : array\PYZus{}like}
         \PY{l+s+sd}{        The linear regression parameters. A vector of shape (n+1, )}
         \PY{l+s+sd}{    }
         \PY{l+s+sd}{    Returns}
         \PY{l+s+sd}{    \PYZhy{}\PYZhy{}\PYZhy{}\PYZhy{}\PYZhy{}\PYZhy{}\PYZhy{}}
         \PY{l+s+sd}{    J : float}
         \PY{l+s+sd}{        The value of the cost function. }
         \PY{l+s+sd}{    }
         \PY{l+s+sd}{    Instructions}
         \PY{l+s+sd}{    \PYZhy{}\PYZhy{}\PYZhy{}\PYZhy{}\PYZhy{}\PYZhy{}\PYZhy{}\PYZhy{}\PYZhy{}\PYZhy{}\PYZhy{}\PYZhy{}}
         \PY{l+s+sd}{    Compute the cost of a particular choice of theta. You should set J to the cost.}
         \PY{l+s+sd}{    \PYZdq{}\PYZdq{}\PYZdq{}}
             \PY{c+c1}{\PYZsh{} Initialize some useful values}
             \PY{n}{m} \PY{o}{=} \PY{n}{y}\PY{o}{.}\PY{n}{shape}\PY{p}{[}\PY{l+m+mi}{0}\PY{p}{]} \PY{c+c1}{\PYZsh{} number of training examples}
             
             \PY{c+c1}{\PYZsh{} You need to return the following variable correctly}
             \PY{n}{J} \PY{o}{=} \PY{l+m+mi}{0}
             
             \PY{c+c1}{\PYZsh{} ======================= YOUR CODE HERE ===========================}
             \PY{n}{temp} \PY{o}{=} \PY{n}{np}\PY{o}{.}\PY{n}{dot}\PY{p}{(}\PY{n}{X}\PY{p}{,} \PY{n}{theta}\PY{p}{)} \PY{o}{\PYZhy{}} \PY{n}{y}
             \PY{n}{J} \PY{o}{=} \PY{n}{np}\PY{o}{.}\PY{n}{sum}\PY{p}{(}\PY{n}{np}\PY{o}{.}\PY{n}{power}\PY{p}{(}\PY{n}{temp}\PY{p}{,} \PY{l+m+mi}{2}\PY{p}{)}\PY{p}{)} \PY{o}{/} \PY{p}{(}\PY{l+m+mi}{2} \PY{o}{*} \PY{n}{m}\PY{p}{)}
             
             \PY{c+c1}{\PYZsh{} ==================================================================}
             \PY{k}{return} \PY{n}{J}
\end{Verbatim}

    \emph{You should now submit your solutions.}

    \begin{Verbatim}[commandchars=\\\{\}]
{\color{incolor}In [{\color{incolor}32}]:} \PY{n}{grader}\PY{p}{[}\PY{l+m+mi}{5}\PY{p}{]} \PY{o}{=} \PY{n}{computeCostMulti}
         \PY{n}{grader}\PY{o}{.}\PY{n}{grade}\PY{p}{(}\PY{p}{)}
\end{Verbatim}

    \begin{Verbatim}[commandchars=\\\{\}]

Submitting Solutions | Programming Exercise linear-regression


    \end{Verbatim}

    \begin{Verbatim}[commandchars=\\\{\}]
Use token from last successful submission (miken024@live.com)? (Y/n):  y

    \end{Verbatim}

    \begin{Verbatim}[commandchars=\\\{\}]
                                  Part Name |     Score | Feedback
                                  --------- |     ----- | --------
                           Warm up exercise |  10 /  10 | Nice work!
          Computing Cost (for one variable) |  40 /  40 | Nice work!
        Gradient Descent (for one variable) |  50 /  50 | Nice work!
                      Feature Normalization |   0 /   0 | Nice work!
    Computing Cost (for multiple variables) |   0 /   0 | Nice work!
  Gradient Descent (for multiple variables) |   0 /   0 | 
                           Normal Equations |   0 /   0 | 
                                  --------------------------------
                                            | 100 / 100 |  


    \end{Verbatim}

    

    \begin{Verbatim}[commandchars=\\\{\}]
{\color{incolor}In [{\color{incolor}34}]:} \PY{k}{def} \PY{n+nf}{gradientDescentMulti}\PY{p}{(}\PY{n}{X}\PY{p}{,} \PY{n}{y}\PY{p}{,} \PY{n}{theta}\PY{p}{,} \PY{n}{alpha}\PY{p}{,} \PY{n}{num\PYZus{}iters}\PY{p}{)}\PY{p}{:}
             \PY{l+s+sd}{\PYZdq{}\PYZdq{}\PYZdq{}}
         \PY{l+s+sd}{    Performs gradient descent to learn theta.}
         \PY{l+s+sd}{    Updates theta by taking num\PYZus{}iters gradient steps with learning rate alpha.}
         \PY{l+s+sd}{        }
         \PY{l+s+sd}{    Parameters}
         \PY{l+s+sd}{    \PYZhy{}\PYZhy{}\PYZhy{}\PYZhy{}\PYZhy{}\PYZhy{}\PYZhy{}\PYZhy{}\PYZhy{}\PYZhy{}}
         \PY{l+s+sd}{    X : array\PYZus{}like}
         \PY{l+s+sd}{        The dataset of shape (m x n+1).}
         \PY{l+s+sd}{    }
         \PY{l+s+sd}{    y : array\PYZus{}like}
         \PY{l+s+sd}{        A vector of shape (m, ) for the values at a given data point.}
         \PY{l+s+sd}{    }
         \PY{l+s+sd}{    theta : array\PYZus{}like}
         \PY{l+s+sd}{        The linear regression parameters. A vector of shape (n+1, )}
         \PY{l+s+sd}{    }
         \PY{l+s+sd}{    alpha : float}
         \PY{l+s+sd}{        The learning rate for gradient descent. }
         \PY{l+s+sd}{    }
         \PY{l+s+sd}{    num\PYZus{}iters : int}
         \PY{l+s+sd}{        The number of iterations to run gradient descent. }
         \PY{l+s+sd}{    }
         \PY{l+s+sd}{    Returns}
         \PY{l+s+sd}{    \PYZhy{}\PYZhy{}\PYZhy{}\PYZhy{}\PYZhy{}\PYZhy{}\PYZhy{}}
         \PY{l+s+sd}{    theta : array\PYZus{}like}
         \PY{l+s+sd}{        The learned linear regression parameters. A vector of shape (n+1, ).}
         \PY{l+s+sd}{    }
         \PY{l+s+sd}{    J\PYZus{}history : list}
         \PY{l+s+sd}{        A python list for the values of the cost function after each iteration.}
         \PY{l+s+sd}{    }
         \PY{l+s+sd}{    Instructions}
         \PY{l+s+sd}{    \PYZhy{}\PYZhy{}\PYZhy{}\PYZhy{}\PYZhy{}\PYZhy{}\PYZhy{}\PYZhy{}\PYZhy{}\PYZhy{}\PYZhy{}\PYZhy{}}
         \PY{l+s+sd}{    Peform a single gradient step on the parameter vector theta.}
         
         \PY{l+s+sd}{    While debugging, it can be useful to print out the values of }
         \PY{l+s+sd}{    the cost function (computeCost) and gradient here.}
         \PY{l+s+sd}{    \PYZdq{}\PYZdq{}\PYZdq{}}
             \PY{c+c1}{\PYZsh{} Initialize some useful values}
             \PY{n}{m} \PY{o}{=} \PY{n}{y}\PY{o}{.}\PY{n}{shape}\PY{p}{[}\PY{l+m+mi}{0}\PY{p}{]} \PY{c+c1}{\PYZsh{} number of training examples}
             
             \PY{c+c1}{\PYZsh{} make a copy of theta, which will be updated by gradient descent}
             \PY{n}{theta} \PY{o}{=} \PY{n}{theta}\PY{o}{.}\PY{n}{copy}\PY{p}{(}\PY{p}{)}
             
             \PY{n}{J\PYZus{}history} \PY{o}{=} \PY{p}{[}\PY{p}{]}
             
             \PY{k}{for} \PY{n}{i} \PY{o+ow}{in} \PY{n+nb}{range}\PY{p}{(}\PY{n}{num\PYZus{}iters}\PY{p}{)}\PY{p}{:}
                 \PY{c+c1}{\PYZsh{} ======================= YOUR CODE HERE ==========================}
                 \PY{n}{theta} \PY{o}{=} \PY{n}{theta} \PY{o}{\PYZhy{}} \PY{n}{alpha} \PY{o}{*} \PY{p}{(}\PY{l+m+mi}{1}\PY{o}{/}\PY{n}{m}\PY{p}{)} \PY{o}{*} \PY{n}{np}\PY{o}{.}\PY{n}{transpose}\PY{p}{(}\PY{n}{X}\PY{p}{)}\PY{o}{.}\PY{n}{dot}\PY{p}{(}\PY{n}{X}\PY{o}{.}\PY{n}{dot}\PY{p}{(}\PY{n}{theta}\PY{p}{)} \PY{o}{\PYZhy{}} \PY{n}{np}\PY{o}{.}\PY{n}{transpose}\PY{p}{(}\PY{n}{y}\PY{p}{)}\PY{p}{)}
                 
                 \PY{c+c1}{\PYZsh{} =================================================================}
                 
                 \PY{c+c1}{\PYZsh{} save the cost J in every iteration}
                 \PY{n}{J\PYZus{}history}\PY{o}{.}\PY{n}{append}\PY{p}{(}\PY{n}{computeCostMulti}\PY{p}{(}\PY{n}{X}\PY{p}{,} \PY{n}{y}\PY{p}{,} \PY{n}{theta}\PY{p}{)}\PY{p}{)}
             
             \PY{k}{return} \PY{n}{theta}\PY{p}{,} \PY{n}{J\PYZus{}history}
\end{Verbatim}

    \emph{You should now submit your solutions.}

    \begin{Verbatim}[commandchars=\\\{\}]
{\color{incolor}In [{\color{incolor}35}]:} \PY{n}{grader}\PY{p}{[}\PY{l+m+mi}{6}\PY{p}{]} \PY{o}{=} \PY{n}{gradientDescentMulti}
         \PY{n}{grader}\PY{o}{.}\PY{n}{grade}\PY{p}{(}\PY{p}{)}
\end{Verbatim}

    \begin{Verbatim}[commandchars=\\\{\}]

Submitting Solutions | Programming Exercise linear-regression


    \end{Verbatim}

    \begin{Verbatim}[commandchars=\\\{\}]
Use token from last successful submission (miken024@live.com)? (Y/n):  y

    \end{Verbatim}

    \begin{Verbatim}[commandchars=\\\{\}]
                                  Part Name |     Score | Feedback
                                  --------- |     ----- | --------
                           Warm up exercise |  10 /  10 | Nice work!
          Computing Cost (for one variable) |  40 /  40 | Nice work!
        Gradient Descent (for one variable) |  50 /  50 | Nice work!
                      Feature Normalization |   0 /   0 | Nice work!
    Computing Cost (for multiple variables) |   0 /   0 | Nice work!
  Gradient Descent (for multiple variables) |   0 /   0 | Nice work!
                           Normal Equations |   0 /   0 | 
                                  --------------------------------
                                            | 100 / 100 |  


    \end{Verbatim}

    \paragraph{3.2.1 Optional (ungraded) exercise: Selecting learning
rates}\label{optional-ungraded-exercise-selecting-learning-rates}

In this part of the exercise, you will get to try out different learning
rates for the dataset and find a learning rate that converges quickly.
You can change the learning rate by modifying the following code and
changing the part of the code that sets the learning rate.

Use your implementation of \texttt{gradientDescentMulti} function and
run gradient descent for about 50 iterations at the chosen learning
rate. The function should also return the history of \(J(\theta)\)
values in a vector \(J\).

After the last iteration, plot the J values against the number of the
iterations.

If you picked a learning rate within a good range, your plot look
similar as the following Figure.

\begin{figure}
\centering
\includegraphics{Figures/learning_rate.png}
\caption{}
\end{figure}

If your graph looks very different, especially if your value of
\(J(\theta)\) increases or even blows up, adjust your learning rate and
try again. We recommend trying values of the learning rate \(\alpha\) on
a log-scale, at multiplicative steps of about 3 times the previous value
(i.e., 0.3, 0.1, 0.03, 0.01 and so on). You may also want to adjust the
number of iterations you are running if that will help you see the
overall trend in the curve.

\textbf{Implementation Note:} If your learning rate is too large,
\(J(\theta)\) can diverge and `blow up', resulting in values which are
too large for computer calculations. In these situations, \texttt{numpy}
will tend to return NaNs. NaN stands for `not a number' and is often
caused by undefined operations that involve −∞ and +∞.

\textbf{MATPLOTLIB tip:} To compare how different learning learning
rates affect convergence, it is helpful to plot \(J\) for several
learning rates on the same figure. This can be done by making
\texttt{alpha} a python list, and looping across the values within this
list, and calling the plot function in every iteration of the loop. It
is also useful to have a legend to distinguish the different lines
within the plot. Search online for \texttt{pyplot.legend} for help on
showing legends in \texttt{matplotlib}.

Notice the changes in the convergence curves as the learning rate
changes. With a small learning rate, you should find that gradient
descent takes a very long time to converge to the optimal value.
Conversely, with a large learning rate, gradient descent might not
converge or might even diverge! Using the best learning rate that you
found, run the script to run gradient descent until convergence to find
the final values of \(\theta\). Next, use this value of \(\theta\) to
predict the price of a house with 1650 square feet and 3 bedrooms. You
will use value later to check your implementation of the normal
equations. Don't forget to normalize your features when you make this
prediction!

    \begin{Verbatim}[commandchars=\\\{\}]
{\color{incolor}In [{\color{incolor}60}]:} \PY{l+s+sd}{\PYZdq{}\PYZdq{}\PYZdq{}}
         \PY{l+s+sd}{Instructions}
         \PY{l+s+sd}{\PYZhy{}\PYZhy{}\PYZhy{}\PYZhy{}\PYZhy{}\PYZhy{}\PYZhy{}\PYZhy{}\PYZhy{}\PYZhy{}\PYZhy{}\PYZhy{}}
         \PY{l+s+sd}{We have provided you with the following starter code that runs}
         \PY{l+s+sd}{gradient descent with a particular learning rate (alpha). }
         
         \PY{l+s+sd}{Your task is to first make sure that your functions \PYZhy{} `computeCost`}
         \PY{l+s+sd}{and `gradientDescent` already work with  this starter code and}
         \PY{l+s+sd}{support multiple variables.}
         
         \PY{l+s+sd}{After that, try running gradient descent with different values of}
         \PY{l+s+sd}{alpha and see which one gives you the best result.}
         
         \PY{l+s+sd}{Finally, you should complete the code at the end to predict the price}
         \PY{l+s+sd}{of a 1650 sq\PYZhy{}ft, 3 br house.}
         
         \PY{l+s+sd}{Hint}
         \PY{l+s+sd}{\PYZhy{}\PYZhy{}\PYZhy{}\PYZhy{}}
         \PY{l+s+sd}{At prediction, make sure you do the same feature normalization.}
         \PY{l+s+sd}{\PYZdq{}\PYZdq{}\PYZdq{}}
         \PY{c+c1}{\PYZsh{} Choose some alpha value \PYZhy{} change this}
         \PY{n}{alpha} \PY{o}{=} \PY{l+m+mf}{0.1}
         \PY{n}{num\PYZus{}iters} \PY{o}{=} \PY{l+m+mi}{400}
         
         \PY{c+c1}{\PYZsh{} init theta and run gradient descent}
         \PY{n}{theta} \PY{o}{=} \PY{n}{np}\PY{o}{.}\PY{n}{zeros}\PY{p}{(}\PY{l+m+mi}{3}\PY{p}{)}
         \PY{n}{theta}\PY{p}{,} \PY{n}{J\PYZus{}history} \PY{o}{=} \PY{n}{gradientDescentMulti}\PY{p}{(}\PY{n}{X}\PY{p}{,} \PY{n}{y}\PY{p}{,} \PY{n}{theta}\PY{p}{,} \PY{n}{alpha}\PY{p}{,} \PY{n}{num\PYZus{}iters}\PY{p}{)}
         
         \PY{c+c1}{\PYZsh{} Plot the convergence graph}
         \PY{n}{pyplot}\PY{o}{.}\PY{n}{plot}\PY{p}{(}\PY{n}{np}\PY{o}{.}\PY{n}{arange}\PY{p}{(}\PY{n+nb}{len}\PY{p}{(}\PY{n}{J\PYZus{}history}\PY{p}{)}\PY{p}{)}\PY{p}{,} \PY{n}{J\PYZus{}history}\PY{p}{,} \PY{n}{lw}\PY{o}{=}\PY{l+m+mi}{2}\PY{p}{)}
         \PY{n}{pyplot}\PY{o}{.}\PY{n}{xlabel}\PY{p}{(}\PY{l+s+s1}{\PYZsq{}}\PY{l+s+s1}{Number of iterations}\PY{l+s+s1}{\PYZsq{}}\PY{p}{)}
         \PY{n}{pyplot}\PY{o}{.}\PY{n}{ylabel}\PY{p}{(}\PY{l+s+s1}{\PYZsq{}}\PY{l+s+s1}{Cost J}\PY{l+s+s1}{\PYZsq{}}\PY{p}{)}
         
         \PY{c+c1}{\PYZsh{} Display the gradient descent\PYZsq{}s result}
         \PY{n+nb}{print}\PY{p}{(}\PY{l+s+s1}{\PYZsq{}}\PY{l+s+s1}{theta computed from gradient descent: }\PY{l+s+si}{\PYZob{}:s\PYZcb{}}\PY{l+s+s1}{\PYZsq{}}\PY{o}{.}\PY{n}{format}\PY{p}{(}\PY{n+nb}{str}\PY{p}{(}\PY{n}{theta}\PY{p}{)}\PY{p}{)}\PY{p}{)}
         
         \PY{c+c1}{\PYZsh{} Estimate the price of a 1650 sq\PYZhy{}ft, 3 br house}
         \PY{c+c1}{\PYZsh{} ======================= YOUR CODE HERE ===========================}
         \PY{c+c1}{\PYZsh{} Recall that the first column of X is all\PYZhy{}ones. }
         \PY{c+c1}{\PYZsh{} Thus, it does not need to be normalized.}
         
         \PY{n}{price} \PY{o}{=} \PY{n}{computeCostMulti}\PY{p}{(}\PY{n}{X}\PY{p}{,} \PY{n}{y}\PY{p}{,} \PY{n}{theta}\PY{p}{)}   \PY{c+c1}{\PYZsh{} You should change this}
         
         \PY{c+c1}{\PYZsh{} ===================================================================}
         
         \PY{n+nb}{print}\PY{p}{(}\PY{l+s+s1}{\PYZsq{}}\PY{l+s+s1}{Predicted price of a 1650 sq\PYZhy{}ft, 3 br house (using gradient descent): \PYZdl{}}\PY{l+s+si}{\PYZob{}:.0f\PYZcb{}}\PY{l+s+s1}{\PYZsq{}}\PY{o}{.}\PY{n}{format}\PY{p}{(}\PY{n}{price}\PY{p}{)}\PY{p}{)}
\end{Verbatim}

    \begin{Verbatim}[commandchars=\\\{\}]
theta computed from gradient descent: [340412.65957447 109447.79558639  -6578.3539709 ]
Predicted price of a 1650 sq-ft, 3 br house (using gradient descent): \$2043280051

    \end{Verbatim}

    \begin{center}
    \adjustimage{max size={0.9\linewidth}{0.9\paperheight}}{output_55_1.png}
    \end{center}
    { \hspace*{\fill} \\}
    
    \emph{You do not need to submit any solutions for this optional
(ungraded) part.}

     \#\#\# 3.3 Normal Equations

In the lecture videos, you learned that the closed-form solution to
linear regression is

\[ \theta = \left( X^T X\right)^{-1} X^T\vec{y}\]

Using this formula does not require any feature scaling, and you will
get an exact solution in one calculation: there is no ``loop until
convergence'' like in gradient descent.

First, we will reload the data to ensure that the variables have not
been modified. Remember that while you do not need to scale your
features, we still need to add a column of 1's to the \(X\) matrix to
have an intercept term (\(\theta_0\)). The code in the next cell will
add the column of 1's to X for you.

    \begin{Verbatim}[commandchars=\\\{\}]
{\color{incolor}In [{\color{incolor}61}]:} \PY{c+c1}{\PYZsh{} Load data}
         \PY{n}{data} \PY{o}{=} \PY{n}{np}\PY{o}{.}\PY{n}{loadtxt}\PY{p}{(}\PY{n}{os}\PY{o}{.}\PY{n}{path}\PY{o}{.}\PY{n}{join}\PY{p}{(}\PY{l+s+s1}{\PYZsq{}}\PY{l+s+s1}{Data}\PY{l+s+s1}{\PYZsq{}}\PY{p}{,} \PY{l+s+s1}{\PYZsq{}}\PY{l+s+s1}{ex1data2.txt}\PY{l+s+s1}{\PYZsq{}}\PY{p}{)}\PY{p}{,} \PY{n}{delimiter}\PY{o}{=}\PY{l+s+s1}{\PYZsq{}}\PY{l+s+s1}{,}\PY{l+s+s1}{\PYZsq{}}\PY{p}{)}
         \PY{n}{X} \PY{o}{=} \PY{n}{data}\PY{p}{[}\PY{p}{:}\PY{p}{,} \PY{p}{:}\PY{l+m+mi}{2}\PY{p}{]}
         \PY{n}{y} \PY{o}{=} \PY{n}{data}\PY{p}{[}\PY{p}{:}\PY{p}{,} \PY{l+m+mi}{2}\PY{p}{]}
         \PY{n}{m} \PY{o}{=} \PY{n}{y}\PY{o}{.}\PY{n}{size}
         \PY{n}{X} \PY{o}{=} \PY{n}{np}\PY{o}{.}\PY{n}{concatenate}\PY{p}{(}\PY{p}{[}\PY{n}{np}\PY{o}{.}\PY{n}{ones}\PY{p}{(}\PY{p}{(}\PY{n}{m}\PY{p}{,} \PY{l+m+mi}{1}\PY{p}{)}\PY{p}{)}\PY{p}{,} \PY{n}{X}\PY{p}{]}\PY{p}{,} \PY{n}{axis}\PY{o}{=}\PY{l+m+mi}{1}\PY{p}{)}
\end{Verbatim}

    Complete the code for the function \texttt{normalEqn} below to use the
formula above to calculate \(\theta\).

    \begin{Verbatim}[commandchars=\\\{\}]
{\color{incolor}In [{\color{incolor}69}]:} \PY{k}{def} \PY{n+nf}{normalEqn}\PY{p}{(}\PY{n}{X}\PY{p}{,} \PY{n}{y}\PY{p}{)}\PY{p}{:}
             \PY{l+s+sd}{\PYZdq{}\PYZdq{}\PYZdq{}}
         \PY{l+s+sd}{    Computes the closed\PYZhy{}form solution to linear regression using the normal equations.}
         \PY{l+s+sd}{    }
         \PY{l+s+sd}{    Parameters}
         \PY{l+s+sd}{    \PYZhy{}\PYZhy{}\PYZhy{}\PYZhy{}\PYZhy{}\PYZhy{}\PYZhy{}\PYZhy{}\PYZhy{}\PYZhy{}}
         \PY{l+s+sd}{    X : array\PYZus{}like}
         \PY{l+s+sd}{        The dataset of shape (m x n+1).}
         \PY{l+s+sd}{    }
         \PY{l+s+sd}{    y : array\PYZus{}like}
         \PY{l+s+sd}{        The value at each data point. A vector of shape (m, ).}
         \PY{l+s+sd}{    }
         \PY{l+s+sd}{    Returns}
         \PY{l+s+sd}{    \PYZhy{}\PYZhy{}\PYZhy{}\PYZhy{}\PYZhy{}\PYZhy{}\PYZhy{}}
         \PY{l+s+sd}{    theta : array\PYZus{}like}
         \PY{l+s+sd}{        Estimated linear regression parameters. A vector of shape (n+1, ).}
         \PY{l+s+sd}{    }
         \PY{l+s+sd}{    Instructions}
         \PY{l+s+sd}{    \PYZhy{}\PYZhy{}\PYZhy{}\PYZhy{}\PYZhy{}\PYZhy{}\PYZhy{}\PYZhy{}\PYZhy{}\PYZhy{}\PYZhy{}\PYZhy{}}
         \PY{l+s+sd}{    Complete the code to compute the closed form solution to linear}
         \PY{l+s+sd}{    regression and put the result in theta.}
         \PY{l+s+sd}{    }
         \PY{l+s+sd}{    Hint}
         \PY{l+s+sd}{    \PYZhy{}\PYZhy{}\PYZhy{}\PYZhy{}}
         \PY{l+s+sd}{    Look up the function `np.linalg.pinv` for computing matrix inverse.}
         \PY{l+s+sd}{    \PYZdq{}\PYZdq{}\PYZdq{}}
             \PY{n}{theta} \PY{o}{=} \PY{n}{np}\PY{o}{.}\PY{n}{zeros}\PY{p}{(}\PY{n}{X}\PY{o}{.}\PY{n}{shape}\PY{p}{[}\PY{l+m+mi}{1}\PY{p}{]}\PY{p}{)}
             
             \PY{c+c1}{\PYZsh{} ===================== YOUR CODE HERE ============================}
         
             \PY{n}{theta} \PY{o}{=} \PY{n}{np}\PY{o}{.}\PY{n}{linalg}\PY{o}{.}\PY{n}{pinv}\PY{p}{(}\PY{n}{np}\PY{o}{.}\PY{n}{transpose}\PY{p}{(}\PY{n}{X}\PY{p}{)}\PY{o}{.}\PY{n}{dot}\PY{p}{(}\PY{n}{X}\PY{p}{)}\PY{p}{)}\PY{o}{.}\PY{n}{dot}\PY{p}{(}\PY{n}{np}\PY{o}{.}\PY{n}{transpose}\PY{p}{(}\PY{n}{X}\PY{p}{)}\PY{o}{.}\PY{n}{dot}\PY{p}{(}\PY{n}{y}\PY{p}{)}\PY{p}{)}
             
             \PY{c+c1}{\PYZsh{} =================================================================}
             \PY{k}{return} \PY{n}{theta}
\end{Verbatim}

    \emph{You should now submit your solutions.}

    \begin{Verbatim}[commandchars=\\\{\}]
{\color{incolor}In [{\color{incolor}64}]:} \PY{n}{grader}\PY{p}{[}\PY{l+m+mi}{7}\PY{p}{]} \PY{o}{=} \PY{n}{normalEqn}
         \PY{n}{grader}\PY{o}{.}\PY{n}{grade}\PY{p}{(}\PY{p}{)}
\end{Verbatim}

    \begin{Verbatim}[commandchars=\\\{\}]

Submitting Solutions | Programming Exercise linear-regression


    \end{Verbatim}

    \begin{Verbatim}[commandchars=\\\{\}]
Use token from last successful submission (miken024@live.com)? (Y/n):  y

    \end{Verbatim}

    \begin{Verbatim}[commandchars=\\\{\}]
                                  Part Name |     Score | Feedback
                                  --------- |     ----- | --------
                           Warm up exercise |  10 /  10 | Nice work!
          Computing Cost (for one variable) |  40 /  40 | Nice work!
        Gradient Descent (for one variable) |  50 /  50 | Nice work!
                      Feature Normalization |   0 /   0 | Nice work!
    Computing Cost (for multiple variables) |   0 /   0 | Nice work!
  Gradient Descent (for multiple variables) |   0 /   0 | Nice work!
                           Normal Equations |   0 /   0 | Nice work!
                                  --------------------------------
                                            | 100 / 100 |  


    \end{Verbatim}

    Optional (ungraded) exercise: Now, once you have found \(\theta\) using
this method, use it to make a price prediction for a 1650-square-foot
house with 3 bedrooms. You should find that gives the same predicted
price as the value you obtained using the model fit with gradient
descent (in Section 3.2.1).

    \begin{Verbatim}[commandchars=\\\{\}]
{\color{incolor}In [{\color{incolor}72}]:} \PY{c+c1}{\PYZsh{} Calculate the parameters from the normal equation}
         \PY{n}{theta} \PY{o}{=} \PY{n}{normalEqn}\PY{p}{(}\PY{n}{X}\PY{p}{,} \PY{n}{y}\PY{p}{)}\PY{p}{;}
         
         \PY{c+c1}{\PYZsh{} Display normal equation\PYZsq{}s result}
         \PY{n+nb}{print}\PY{p}{(}\PY{l+s+s1}{\PYZsq{}}\PY{l+s+s1}{Theta computed from the normal equations: }\PY{l+s+si}{\PYZob{}:s\PYZcb{}}\PY{l+s+s1}{\PYZsq{}}\PY{o}{.}\PY{n}{format}\PY{p}{(}\PY{n+nb}{str}\PY{p}{(}\PY{n}{theta}\PY{p}{)}\PY{p}{)}\PY{p}{)}\PY{p}{;}
         
         \PY{c+c1}{\PYZsh{} Estimate the price of a 1650 sq\PYZhy{}ft, 3 br house}
         \PY{c+c1}{\PYZsh{} ====================== YOUR CODE HERE ======================}
         
         \PY{n}{price} \PY{o}{=} \PY{n}{np}\PY{o}{.}\PY{n}{array}\PY{p}{(}\PY{p}{[}\PY{l+m+mi}{1}\PY{p}{,} \PY{l+m+mi}{1650}\PY{p}{,} \PY{l+m+mi}{3}\PY{p}{]}\PY{p}{)}\PY{o}{.}\PY{n}{dot}\PY{p}{(}\PY{n}{theta}\PY{p}{)} \PY{c+c1}{\PYZsh{} You should change this}
         
         \PY{c+c1}{\PYZsh{} ============================================================}
         
         \PY{n+nb}{print}\PY{p}{(}\PY{l+s+s1}{\PYZsq{}}\PY{l+s+s1}{Predicted price of a 1650 sq\PYZhy{}ft, 3 br house (using normal equations): \PYZdl{}}\PY{l+s+si}{\PYZob{}:.0f\PYZcb{}}\PY{l+s+s1}{\PYZsq{}}\PY{o}{.}\PY{n}{format}\PY{p}{(}\PY{n}{price}\PY{p}{)}\PY{p}{)}
\end{Verbatim}

    \begin{Verbatim}[commandchars=\\\{\}]
[89597.90954361   139.21067402 -8738.01911255]
Theta computed from the normal equations: [89597.90954361   139.21067402 -8738.01911255]
Predicted price of a 1650 sq-ft, 3 br house (using normal equations): \$293081

    \end{Verbatim}

    \begin{Verbatim}[commandchars=\\\{\}]
{\color{incolor}In [{\color{incolor} }]:} 
\end{Verbatim}


    % Add a bibliography block to the postdoc
    
    
    
    \end{document}
